\documentclass[10pt,a4paper]{article}
\usepackage[utf8]{inputenc}
\usepackage{amsmath}
\usepackage{amsfonts}
\usepackage{amssymb}
\usepackage{amsthm}
\author{Alessio Santoro}
\title{Congettura preimmagine di 231}
\date{}
\usepackage{tikz}
\usetikzlibrary{matrix,trees,arrows}

\usepackage{stmaryrd}		% various mathematical symbols

% For tikz pictures
\usetikzlibrary{positioning}
\usetikzlibrary{fit}
\usetikzlibrary{patterns}

\newcommand{\ZZ}{\mathbb{Z}}           % Integers
\newcommand{\sym}{\mathcal{S}}         % Permutations
\DeclareMathOperator{\unSvar}{{\mathsf{cand}}}
\DeclareMathOperator{\inv}{{\mathsf{inv}}}
\DeclareMathOperator{\ninv}{{\mathsf{ninv}}}
\DeclareMathOperator{\symS}{{\mathcal{S}}}

\newcommand{\dbrac}[1]{{\llbracket #1 \rrbracket}} 	% double brackets

\DeclareMathOperator{\Av}{\mathrm{Av}}

%%%%%%%%%%% MACROS FOR DRAWING INTERVAL AND MESH PATTERNS %%%%%%%%%%%

% Sub-macros
\newcommand{\shadetheboxes}[1]{
	\foreach \x/\y in {#1}
      	\fill[pattern color = black!65, pattern=north east lines] (\x,\y) rectangle +(1,1);
	}
	
\newcommand{\drawthegrid}[1]{
	\draw (0.01,0.01) grid (#1+0.99,#1+0.99);
	}
	
\newcommand{\drawtheclpattern}[1]{
	\foreach \x/\y in {#1}
      	\filldraw (\x,\y) circle (6pt);
	}
	
\newcommand{\drawtheclpatternwhite}[1]{
	\foreach \x/\y in {#1}
      	\draw[fill=white] (\x,\y) circle (6pt);
	}
	
\newcommand{\drawtheclpatternwhitebig}[1]{
	\foreach \x/\y in {#1}
		\draw[fill=white] (\x,\y) circle (11pt);
	}
	
\newcommand{\drawspecialbox}[1]{
	\foreach \x/\y/\z/\w/\A in {#1}
		{
       		\fill[color = white!100, opacity=1, rounded corners = 1.5pt] (\x+0.125,\y+0.125) rectangle (\z-0.125,\w-0.125);
       		\draw[color = black, rounded corners = 1.5pt] (\x+0.125,\y+0.125) rectangle (\z-0.125,\w-0.125);
       		\fill[black] (\x/2+\z/2,\y/2+\w/2) node {$\scriptstyle\A$};
       	}
    }
    
\newcommand{\drawspecialboxnomark}[1]{
	\foreach \x/\y/\z/\w in {#1}
		{
       		\fill[color = white!100, opacity=1, rounded corners = 1.5pt] (\x+0.125,\y+0.125) rectangle (\z-0.125,\w-0.125);
       		\draw[color = black, rounded corners = 1.5pt] (\x+0.125,\y+0.125) rectangle (\z-0.125,\w-0.125);
       	}
    }
    
\newcommand{\drawspecialboxthemark}[1]{
	\foreach \x/\y/\z/\w/\A in {#1}
		\fill[black] (\x/2+\z/2,\y/2+\w/2) node {$\scriptstyle\A$};
    }
    
\newcommand{\drawspecialshadedbox}[1]{
	\foreach \x/\y/\z/\w/\A in {#1}
		{
       		\fill[color = white!100, opacity=1, rounded corners=1.5pt] (\x+0.125,\y+0.125) rectangle (\z-0.125,\w-0.125);
       		\fill[pattern color = black!65, pattern=north east lines, rounded corners=1.5pt] (\x+0.125,\y+0.125) rectangle (\z-0.125,\w-0.125);
       		\draw[color = black, rounded corners=1.5pt] (\x+0.125,\y+0.125) rectangle (\z-0.125,\w-0.125);
       		\fill[black] (\x/2+\z/2,\y/2+\w/2) node {$\scriptstyle\A$};
       	}
    }
    

\newcommand{\mpattern}[4]{										% mesh pattern
  \raisebox{0.6ex}{
  \begin{tikzpicture}[scale=0.35, baseline=(current bounding box.center), #1]
  	\useasboundingbox (0.0,-0.1) rectangle (#2+1.4,#2+1.1);
	
    \shadetheboxes{#4}
    
    \drawthegrid{#2}
    
    \drawtheclpattern{#3}
    
  \end{tikzpicture}}
}

\newcommand{\mpatternnl}[4]{										% mesh pattern - nl
  \raisebox{0.6ex}{
  \begin{tikzpicture}[scale=0.35, baseline=(current bounding box.center), #1]
  	\useasboundingbox (0.85,-0.1) rectangle (#2+1.4,#2+1.1);
	
    \shadetheboxes{#4}
    
    \drawtheclpattern{#3}
    
  \end{tikzpicture}}
}

\newcommand{\onetwo}{\mpatternnl{scale=0.2}{2}{1/1,2/2}{}}		% the pattern 12 with no lines

\newcommand{\mpatternww}[5]{									% mesh pattern with white and black dots
  \raisebox{0.6ex}{
  \begin{tikzpicture}[scale=0.35, baseline=(current bounding box.center), #1]
  	\useasboundingbox (0.0,-0.1) rectangle (#2+1.4,#2+1.1);
	
    \shadetheboxes{#5}
    
    \drawthegrid{#2}
    
    \drawtheclpatternwhite{#4}
    \drawtheclpattern{#3}
    
  \end{tikzpicture}}
}

\newcommand{\mpatternwwo}[6]{									% mesh pattern with white and black dots overlapping
  \raisebox{0.6ex}{
  \begin{tikzpicture}[scale=0.35, baseline=(current bounding box.center), #1]
  \useasboundingbox (0.0,-0.1) rectangle (#2+1.4,#2+1.1);
  
    \shadetheboxes{#6}
    
    \drawthegrid{#2}
    
    \drawtheclpatternwhite{#4}
    \drawtheclpatternwhitebig{#5}
    \drawtheclpattern{#3}
    
  \end{tikzpicture}}
}

\newcommand{\mmpattern}[5]{									% mesh patterns with a special box marked
  \raisebox{0.6ex}{
  \begin{tikzpicture}[scale=0.35, baseline=(current bounding box.center), #1]
  \useasboundingbox (0.0,-0.1) rectangle (#2+1.4,#2+1.1);
    
    \shadetheboxes{#4}
    
    \drawthegrid{#2}
    
    \drawspecialbox{#5}
    
    \drawtheclpattern{#3}

  \end{tikzpicture}}
}

\newcommand{\mmpatternwwo}[7]{									% mesh pattern with white and black dots overlapping
  \raisebox{0.6ex}{
  \begin{tikzpicture}[scale=0.35, baseline=(current bounding box.center), #1]
  \useasboundingbox (0.0,-0.1) rectangle (#2+1.4,#2+1.1);
    
    \shadetheboxes{#6}
    
    \drawthegrid{#2}
    
    \drawspecialbox{#7}
    
    \drawtheclpatternwhite{#4}
    \drawtheclpatternwhitebig{#5}
    \drawtheclpattern{#3}
    
  \end{tikzpicture}}
}

\newcommand{\mmpatternloc}[6]{									% mesh patterns with a special box marked
  \raisebox{0.6ex}{											% can control location of mark
  \begin{tikzpicture}[scale=0.35, baseline=(current bounding box.center), #1]
  \useasboundingbox (0.0,-0.1) rectangle (#2+1.4,#2+1.1);
    
    \shadetheboxes{#4}
    
    \drawthegrid{#2}
    
    \drawspecialboxnomark{#5}
    \drawspecialboxthemark{#6}
    
    \drawtheclpattern{#3}

  \end{tikzpicture}}
}

\newcommand{\mmpatternwwoloc}[8]{									% mesh pattern with white and black dots overlapping
  \raisebox{0.6ex}{												% can control location of mark
  \begin{tikzpicture}[scale=0.35, baseline=(current bounding box.center), #1]
  \useasboundingbox (0.0,-0.1) rectangle (#2+1.4,#2+1.1);
    
    \shadetheboxes{#6}
    
    \drawthegrid{#2}
    
    \drawspecialboxnomark{#7}
    \drawspecialboxthemark{#8}
    
    \drawtheclpatternwhite{#4}
    \drawtheclpatternwhitebig{#5}
    \drawtheclpattern{#3}
    
  \end{tikzpicture}}
}

\newcommand{\decpattern}[6]{									% decorated pattern
  \raisebox{0.6ex}{
  \begin{tikzpicture}[scale=0.35, baseline=(current bounding box.center), #1]
  \useasboundingbox (0.0,-0.1) rectangle (#2+1.4,#2+1.1);
    
    \shadetheboxes{#4}
    
    \drawthegrid{#2}
       
    \drawspecialbox{#5}
    \drawspecialshadedbox{#6}
    
    \drawtheclpattern{#3}

  \end{tikzpicture}}
}

\newcommand{\decpatternww}[8]{								% decorated pattern w white
  \raisebox{0.6ex}{
  \begin{tikzpicture}[scale=0.35, baseline=(current bounding box.center), #1]
  \useasboundingbox (0.0,-0.1) rectangle (#2+1.4,#2+1.1);
    
    \shadetheboxes{#6}
    
    \drawthegrid{#2}
       
    \drawspecialbox{#7}
    \drawspecialshadedbox{#8}
    
    \drawtheclpatternwhite{#4}
    \drawtheclpatternwhitebig{#5}
    \drawtheclpattern{#3}

  \end{tikzpicture}}
}

\newcommand{\mpatternlabelled}[5]{										% mesh pattern with labels for some boxes
  \raisebox{0.6ex}{
  \begin{tikzpicture}[scale=0.35, baseline=(current bounding box.center), #1]
    
    \shadetheboxes{#4}
    
    \drawthegrid{#2}
    
    \drawtheclpattern{#3}
    
    \draw (0.5,0.5) -- (0.5,-0.5);
    \fill[black] (0.5,-1) node {$\scriptstyle(0,0)$};
    \draw (0.5,2.5) -- (0.5,4.5);
    \fill[black] (0.5,5) node {$\scriptstyle(0,2)$};
    \draw (1.5,2.5) -- (1.5,6);
    \fill[black] (1.5,6.5) node {$\scriptstyle(1,2)$};
    \draw (2.5,2.5) -- (2.5,4.5);
    \fill[black] (2.5,5) node {$\scriptstyle(2,2)$};
    
  \end{tikzpicture}}
}

\begin{document}
\maketitle
L'idea che si vuole dimostrare è:
$$Cons^{-1}Av(231) = Av(2431, 4231, 23514, \overline{5}3\overline{5}2614)$$
\paragraph*{Importante (nota metodologica):}$\overline{5}3\overline{5}2614$ si può scrivere così o va separato in $3\overline{5}2614, \overline{5}32614$?\\\\
Per dimostrare la congettura occorre dimostrare la doppia inclusione:
\begin{description}
\item[(A)]$Cons^{-1}Av(231) \subseteq Av(2431, 4231, 23514, \overline{5}3\overline{5}2614)$
\item[(B)]$Av(2431, 4231, 23514, \overline{5}3\overline{5}2614) \subseteq Cons^{-1}Av(231)$
\end{description}
\paragraph*{Nota:}Per tutta la dimostrazione i valori che formano i pattern saranno indicati da $a,b,c,d$ ed $e$ (quando presente), $n$ indicherà la lunghezza della permutazione $\pi$ e si intende $1\leq{a}<b<c<d<e\leq{n}$.
\paragraph*{Acluni risutltati utili:} Sia $\pi=\pi_1\pi_2\dots\pi_n$ una permutazione, se durante l'esecuzione di $Cons$ un elemento l'elemento $\pi_i$ provoca un $pop$ tutti gli elementi successivi minori di $\pi_i$ effettuano un bypass.Similmente se $\pi_i$ effettua un bypass tutti gli elementi succesivi minori di $\pi_i$ effettuano un bypass.
\section*{(A) $Cons^{-1}Av(231) \subseteq Av(2431, 4231, 23514, \overline{5}3\overline{5}2614)$}
\begin{proof}
Dobbiamo dimostrare che tutte le permutazioni la cui immagine evita il pattern 231 evitano i pattern $2431, 4231, 23514, 3\overline{5}2614, \overline{5}32614$. Ovvero si dimostrerà che ogni permutazione $\pi$ tale che $231\preceq{Cons}(\pi)$ contiene uno dei pattern elencati.\\
Si osserva che se 3 elementi nell'immagine $Cons(\pi)$ formano un pattern 231, quegli stessi elementi devono formare un pattern 231 o 321 nella preimmagine, dato che \textit{Cons} non produce nuove inversioni. Si analizzano questi due casi separatamente.
\paragraph*{$231\preceq\pi$:}$$\pi=\dots b\dots c\dots a\dots$$$$Cons(\pi)=\dots b\dots c\dots a\dots$$
Se $b$ entra in coda deve uscire prima $c$ possa accodarsi a sua volta. Un elemento $d>c$ provoca un $pop$ e non è possibile che si accodi. $c,a$ effettuano un $bypass$, e nell'output troviamo $bca$. Questo è il pattern 2431.\\
Se $b$ effettua un $bypass$ e $c$ si accoda si deve verificare un $pop$, perchè $c$ deve uscire prima che $a$ bypassi. È necessario un elemento $e>c$ posizionato tra $c$ e $a$.
\begin{center}
\mmpattern{scale=1}{3}{1/2,2/3,3/1}{}{2/3/3/4/1}$\Rightarrow$\mpattern{scale=1}{4}{1/2,2/3,3/4,4/1}{}
\end{center}
Tuttavia occorre assicurarsi che $e$ non si accodi dopo $c$, poichè la coda contiene solo elementi consecutivi, questa condizione è realizzata solo se un elemento $d$ si trova in una posizione successiva a $e$.
\begin{center}
\mpattern{scale=1}{4}{1/2,2/3,3/4,4/1}{}$\Rightarrow$\mmpattern{scale=1}{4}{1/2,2/3,3/4,4/1}{}{3/3/5/4/1}
\end{center}
Queste condizioni sono soddisfatte dai pattern 23541 e 23514.\\
Infine, se è presente un valore $d$ prima di $b$, il pattern 231 dato da $bca$ resta invariato (pattern 4231): se $d$ effettua un bypass o se se provoca un $pop$, tutti gli elementi dopo effettuano un bypass, se $d$ entra in coda $b,c$ può effettuare un $bypass$ o provocare un $pop$, in ogni caso $b$ viene inserito nell'output prima di $c$, e dopo $a$ effettua sicuramente un $bypass$.
\paragraph*{$321\preceq\pi$:}$$\pi=\dots c\dots b\dots a\dots$$$$Cons(\pi)=\dots b\dots c\dots a\dots$$
Deve avvenire che $c$ entri in coda. Quando questo avviene $b$ può solo bypassare, quindi deve avvenire un $pop$ tra $b$ e $a$ e non tra $c$ e $b$.
\begin{center}
\mmpattern{scale=1}{3}{1/3,2/2,3/1}{0/3,1/3}{2/3/3/4/1}$\Rightarrow$\mmpattern{scale=1}{4}{1/3,2/2,3/4,4/1}{0/3,1/3}{}
\end{center}
Si ottiene così un pattern classico 3241 con alcune aree oscurate.\\Come fatto precedentemente si dovrà essere sicuri che $d$ provochi un $pop$ e non si accodi, quindi si aggiungerà un nuovo elemento di valore compreso tra $c$ e $d$ posizionato dopo di $a$:
\begin{center}
\mmpattern{scale=1}{4}{1/3,2/2,3/4,4/1}{0/3,1/3}{3/3/5/4/1}$\Rightarrow$\mmpattern{scale=1}{5}{1/3,2/2,3/5,4/4,5/1}{0/4,1/4}{},\mmpattern{scale=1}{5}{1/3,2/2,3/5,4/1,5/4}{0/4,1/4}{}
\end{center}
I pattern che si ottengono sono $\overline{5}3\overline{5}2614$,$\overline{5}3\overline{5}2641$.\\\\\\
Alcuni risultati che si sono ottenuti possono essere semplificati, dato che l'insieme trovato non è minimo: $\overline{5}3\overline{5}2641$ e 23541 contengono entrambi il pattern 2431.\\
I pattern trovati, una volta semplificati, provano che:
\begin{eqnarray*}
Cons^{-1}Av(231)\subseteq{Av(2431, 4231, 23514, 23541, \overline{5}3\overline{5}2614)}\\ = Av(2431, 4231, 23514, \overline{5}3\overline{5}2614)
\end{eqnarray*}
\end{proof}
\section*{(B) $Av(2431, 4231, 23514, \overline{5}3\overline{5}2614) \subseteq Cons^{-1}Av(231)$}
\begin{proof}
Dobbiamo dimostrare che tutte le permutazioni che non contengono i pattern $2431, 4231, 23514, 3\overline{5}2614, \overline{5}32614$ sono contenute anche nell'insieme \\$Cons^{-1}Av(231)$, ovvero che l'immagine di ogni permutazione che contiene almeno uno di questi pattern contiene un pattern 231:
\subsection*{2431}
$$\pi = \dots{b}\dots{d}\dots{c}\dots{a}\dots$$
Se $b$ provoca un $pop$ questo non influenza gli altri 3 elementi che ci interessano, quindi questa eventualità rientra nel caso in cui $b$ entri in coda.\\Se $b$ entra in coda un elemento l'elemento $d$ provoca un $pop$ (eventualmente altri valori possono provocare $pop$ tra $b$ e $c$), $c$ e $a$ effettuano un $byapss$ e nell'output si ha $231\preceq{bda}\subseteq{Cons(\pi)}$.\\
Se $b$ effettua un $bypass$ lo effettua sempre anche $a$.\\
Se poi $d$ effettua un $bypass$ si ha che $231\preceq{bda}\subseteq{Cons(\pi)}$ (lo stesso vale se anche $c$ effettua un $byapss$).\\
Se $d$ entra in coda $c$ può solo bypassare (caso precedente) o provocare un $pop$, e allora $231\preceq{bda}\subseteq{Cons(\pi)}$ nell'output
\subsection*{4231}
$$\pi = \dots{d}\dots{b}\dots{c}\dots{a}\dots$$
Se $d$ provoca un $pop$ questo non influenza gli altri 3 elementi che ci interessano, quindi questa eventualità rientra nel caso in cui $d$ entri in coda.\\
Se $d$ effettua un $bypass$ tutti gli altri elementi del pattern effettuano un $bypass$ e il pattern 231 dato da $bca$ è presente nell'output.\\
Se $d$ entra in coda tutti gli altri elementi del pattern non possono provocare un $pop$: perchè $c$ provochi un $pop$ deve essere che $Back(Q)=d<c<Front(Q)$ e questo è assurdo perchè la coda contiene elementi consecutivi e $c$ non è in coda (la stessa considerazione vale per $a,b$).Quindi se $d$ entra in coda gli altri elementi del pattern effettuano in $bypass$ e il pattern 231 dato da $bca$ è presente nell'output.
\subsection*{23514}$$\pi = \dots{b}\dots{c}\dots{e}\dots{a}\dots{d}\dots$$Se $b$ provoca un $pop$ questo non influenza gli altri 3 elementi che ci interessano, quindi questa eventualità rientra nel caso in cui $b$ entri in coda.\\
Se $b$ entra in coda $c$ può accodarsi a sua volta o provocare un $pop$. Se $c$ si accoda $e$ provoca un $pop$, $a$ effettua un $bypass$ e nell'immagine è presente il pattern 231 $bca$. Se $c$ provoca un $pop$ $b$ viene inserito nell'output, $e$ provoca un $pop$ ed il caso è analogo al precedente.\\
Se $b, c$ effettuano un $bypass$ lo fa anche $a$, quindi si ha $bca$ nell'immagine.\\
Se solo $b$ effettua un $bypass$ $c$ può accodarsi o provocare un $pop$; in entrambi i casi $e$ (o un altro elemento maggiore di $c$ tra $c$ e $e$) provoca un $pop$ aggiungendo $c$ all'output, $a$ effettua un $bypass$ e si ha $bca$ nell'output.
\subsection*{$\overline{5}$3$\overline{5}$2614}
Il pattern può anche essere rappresentato dal seguente mesh-pattern:
\begin{center}
\mpattern{scale=1}{5}{1/3,2/2,3/5,4/1,5/4}{0/4, 1/4}
\end{center}
Quindi una permutazione che contiene occorrenza di questo pattern ha la forma:
$$\pi = \dots{c}\dots{b}\dots{e}\dots{a}\dots{d}\dots$$dove nessun valore precedente a $b$ è compreso tra $d$ e $e$.\\\\
Se $c$ provoca un $pop$ il caso è analogo a quello in cui si accodi e basta.\\
Se $c$ si accoda $b$ effettua un $bypass$ e $e$ (o eventualmente anche un altro elemento maggiore di $b$ tra $b$ e $e$) provoca un $pop$, in seguito $a$ effettua un $bypass$ e nell'immagine si trova la sequenza $bca$.\\
Se $c$ effettua un $bypass$ lo fa anche $b$. Se anche $e$ effettua un $bypass$ si ha un pattern 231 nella sequenza $cea$ dopo il $bypass$ di $a$. Se invece $e$ provoca un $pop$ gli elementi estratti sono sicuramente maggiori di $c$: $c$ ha effettuato un $bypass$ quando doveva valere $c<Front(Q)=c^+$, in seguito $a$ effettua un $bypass$ e si ha che l'immagine contiene $cc^+a$.\\
Se $e$ si accoda deve esserci stato almento un elemento $d^+:d<d^+<e$, tra $b$ e $e$ (a causa delle componenti barrate del pattern). Dal momento che la coda prima di incontrare $d^+$ contiene solo membri strettamente minori di $d$, $d^+$ provoca un $pop$ e si applicano a $d^+$ le stesse considerazioni di $e$ nel caso in cui effettui un $pop$.\\\\
\end{proof}
\end{document}