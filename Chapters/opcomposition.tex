\chapter{Composizione di operatori di ordinamento}
\'E molto interessante studiare la combinazione dei vari operatori e le relazioni tra determinati pattern e le loro preimmagini.\\
\paragraph*{Preimmagini} Sia $X$ un operatore di ordinamento, la preimmagine di un certo pattern $p$, secondo $X$, indicata con $X^{-1}(p)$ rappresenta l'insieme di tutte le permutazioni la cui immagine secondo l'operatore $X$ contiene il pattern $p$.$$X^{-1}(p) = \{\beta : p\preceq X(\beta)\}$$\\
Dato che $Av(21)$ \'e l'insieme formato dalle sole permutazioni identit\'a, dato che contiene tutte le permutazioni in cui nessun elemento sia disordinato rispetto ad un altro, ovvero solo le permutazioni incrementali, si indica con $X^{-1}(Av(21))$ l'insieme di tutte le permutazioni ordinabili da $X$.\\\\
Ad esempio, dato che \'e noto che l'operatore \textit{bubblesort} ordina solo le permutazioni che non contengono pattern 231 e 321, vale che:$$B^{-1}(Av(21)) = Av(231,321)$$
\paragraph*{Composizione di operatori} Siano due operatori di ordinamento $X$ e $Y$, la loro composizione \'e indicata con $( XY ) = ( X \circ Y )$, qundi, ad esempio, la composizione di \textit{Stacksort} e \textit{Bubblesort} (in questo ordine) si indica con $SB(\pi) = (S \circ B)(\pi) = S(B(\pi))$.
\paragraph*{Algoritmi per il calcolo di preimmagini} Sono gi\'a stati prodotti alcuni algoritmi per calcolare le preimmagini di pattern secondo gli operaratori \textit{bubblesort}\cite{albert2010inverse}, \textit{stacksort}\cite{claesson2012sorting} e \textit{queuesort}\cite{magnusson2013sorting}\cite{cioni2021characterization}.\\
Questo ci permette, quando si combinano due operatori di ordinamento, di cercare per quali pattern l'operatore che viene applicato per primo produce permutazioni che siano ordinabili dal secondo.
\paragraph*{Combinazione di \textit{Stacksort} e \textit{Bubblesort}} Si considera dunque la composione $SB = S\circ{B}$ e ci si chiede quali permutazioni possano essere ordinate da esso.$$(SB)^{-1}=B^{-1}S^{-1}(Av(21))=B^{-1}(Av(231))$$
Dunque ricercando per quali permutazioni \textit{bubblesort} evita il pattern 231 si trovano le condizioni per cui una permutazione risulta ordinabile da \textit{SB}.\\
Utilizzando i suddetti algoritmi\cite{albert2010inverse} si ottiene che:$$(SB)^{-1}(Av(21))=Av(3241, 2341, 4231, 2431)$$
Si pu\'o notare che bubble sort, applicato a tutte le combinazioni trovate, produce sempre un pattern 231:
\begin{description}
\item[2341:] viene scambiato il 4 con l'1, ottenndo 2314
\item[2431:] viene scambiato il 4 con il 3, poi con l'1, ottenndo 2314
\item[3241:] vengono scambiati il 2 con il 3 e il 4 con l'1, ottenendo 2341
\item[4231:] viene scambiato il 4 con tutti gli elementi fino ad essere alla fine della sequenza.
\end{description}
Si nota da questo esempio un'approccio (senz'altro meno rigoroso degli algoritmi gi\'a formalizzati) che permette in generale di cercare preimmagini secondo bubblesort: dal pattern che si vuole ottenere si cercano le coppie di valori non ordinate (che quindi dovranno essere disordinate anche prima della passata) e si cercano le sequenze che le contengono, queste sequenze verranno dette \textbf{candidati} e ogni preimmagine del pattern deve contenere almeno uno di questi.\\Nel caso specifico del bubblesort, posizionando prima di una coppia disordinata nel pattern originale un valore maggiore, si evita che questa venga ordinata e questo ci permette di trovare le possibili preimmaigini valutando le possibili posizioni e i possibili valori che questo pu\'o assumere.
\paragraph*{Esempio di calcolo di una preimmagine}Volendo ricercare le preimmagini di $231$ secondo \textit{bubblesort} si osserva che le coppie disordinate in $231$ sono $(2,1),(3,1)$, da questo si ricavano una lista di candidati, ovvero di pattern contenuti nelle preimmagini, che dovranno comprendere almeno le stesse coppie disordinate: ovvero $231$ e $321$.
\begin{description}
\item[231:] \textit{bubblesort} lascia invariata la coppia $(2,3)$, ma per produrre un pattern $231$ \'e necessario che non venga scambiata la coppia $(3,1)$, per fare questo si posiziona il valore $4$ prima dell'$1$, in qualunque posizione, ottenendo $4231$, $2431$, $2341$;
\begin{center}
\mmpattern{scale=1.4}{3}{1/2,2/3,3/1}{}{0/3/3/4/1}
\end{center}
\item[321:] per ottenere un $231$ \'e necessario che venga scambiata la coppia $(3,2)$ ma non $(3,1)$, dunque \'e necessario che un valore maggiore di $3$ sia posizionato dopo $2$ ma prima di $1$, cos\'i si ottiene $3241$.
\begin{center}
\mmpattern{scale=1.4}{3}{1/3,2/2,3/1}{}{2/3/3/4/1}
\end{center}
\end{description}
Da queste considerazioni si pu\'o verificare il risultato ottenuto prima per $SB$.\\\\
Lo stesso metodo pu\'o essere applicato alla combinazione di \textit{queuesort} e \textit{bubblesort}:
\section*{$\textit{Q}\circ\textit{B}$:}$$(QB)^{-1}(Av(21))=B^{-1}Q^{-1}(Av(21))=B^{-1}(Av(321))$$Si osserva che le coppie non invertite in $321$ sono $(3,2),(3,1),(2,1)$ e l'unico pattern minimo che le contiene tutte \'e appunto $321$.\\Per fare in modo che \textit{bubblesort} produca un $321$ bisogna dunque che sia presente un valore $4$ prima di $2$: sia che sia posizionato prima di $3$ o tra $3$ e $2$, esso \'e l'unico elemento (tra quelli che compongono il pattern) che viene spostato e il pattern $321$ presente nell'input \'e ancora presente nell'output.\\
\begin{center}
\mmpattern{scale=1.4}{3}{1/3,2/2,3/1}{}{0/3/2/4/1}
\end{center}
Cos\'i si ottengono le preimmagini $4321,3421$.$$QB^{-1}(Av(21))=Av(4321,3421)$$
\section*{Composizioni che terminano con \textit{stacksort}}
L'algoritmo per le preimmagini di \textit{stacksort} \'e abbstanza simile a quello per \textit{bubblesort}.\\
Anche in questo caso si osservano tutte le coppie disordinate contenute nell'immagine in esame e se ne tre una lista di pattern minimi candidati ad essere preimmagini e li si esaminano ad uno ad uno.\\
Per ogni coppia disordinata $(b,a)$ presente nel candidato: 
\begin{description}
\item se $(b,a)$ \'e presente anche nell'immagine allora deve essere presente un elemento $c>b$ tra $b$ e $a$ che fa uscire $b$ dalla pila prima che $a$ vi entri
\begin{center}
\mmpattern{scale=1}{2}{1/2,2/1}{}{} $\Longrightarrow$\mmpattern{scale=1}{2}{1/2,2/1}{}{1/2/2/3/1} 
\end{center}
\item se invece $(b,a)$ non \'e presente allora si pu\'o escludere la presenza di tale elemento
\begin{center}
\mmpattern{scale=1}{2}{1/2,2/1}{}{} $\Longrightarrow$\mmpattern{scale=1}{2}{1/2,2/1}{1/2}{} 
\end{center}
\end{description}
\subsection*{$\textit{Q}\circ\textit{S}$}$$(QS)^{-1}(Av(21))=S^{-1}Q^{-1}(Av(21))=S^{-1}(Av(321))$$
Si ricerca dunque le preimmagini di $321$ secondo \textit{stacksort}.\\$321$ \'e l'unico candidato, quindi sappiamo che tutte le preimmagini devono contenere il pattern $321$.\\
L'elemento $3$ deve entrare nella pila ed uscirne prima del $2$, per fare ci\'o deve esserci un elemento $3^+$ maggiore di $3$ tra $3$ e $2$. Similmente \'e necessario un elemento $2^+>2$ tra $2$ e $1$ per assicurare che $2$ esca dalla pila prima che $1$ vi entri.\\
\begin{center}
\mmpattern{scale=1.5}{3}{1/3,2/2,3/1}{}{1/3/2/4/1,2/2/3/4/1}
\end{center}
Dunque le preimmagini che cerchiamo devono essere nella forma $33^+22^+1$.\\
Se $2^+<3$ allora la preimmagine assume la forma del pattern $45231$.\\
Altrimenti se $2^+>3$, si possono ottenere due diverse preimmagini:
\begin{description}
	\item[$3^+>2^+$] genera il pattern $35241$
	\item[$2^+>3^+$] genera il pattern $34251$
\end{description}
$$(QS)^{-1}(Av(21)) = Av(34251, 35241, 45231)$$
\subsection*{$\textit{B}\circ\textit{S}$}$$(BS)^{-1}(Av(21))=S^{-1}B^{-1}(Av(21))=S^{-1}(Av(231,321))$$
Gi\'a dall'analisi della combinazione precedente \'e risultato che $S^{-1}(Av(321))=Av(34251, 35241, 45231)$ qundi \'e necessario calcolare solo $S^{-1}(Av(231))$. Quest'ultimo risultato \'e stato ampiamente studiato in letteratura, in quanto analogo al caso di una variante di \textit{stacksort} che utilizza 2 pile. Il risultato che si ottiene \'e dunque che $S^{-1}(Av(231))=Av(2341, 3\overline{5}241)$\cite{claesson2012sorting}.\\\\
Seguendo l'algoritmo applicato finora si osserva che i pattern candidati per le preimmagini sono $231,321$.\\\\
Per ottenere con \textit{stacksort} un pattern $231$ da un pattern $321$ \'e necessario che il $3$ entri pila e che vi rimanga fino all'ingresso di $2$, il che esclude la presenza di valori maggiori di $3$ tra $3$ e $2$, tuttavia dato che entrambi devono uscire dalla pila prima di $1$ deve esserci almeno un valore maggiore di $3$ tra $2$ e $1$: i pattern che implicano queste condizioni sono $3\overline{4}251$ e $3\overline{5}241$; si osserva che se i valori maggiori di $3$ sono ordinati tra loro quando si arriva al secondo il primo viene estratto, contraddicendo le condizioni, dunque l'unico pattern minimo che contiene un pattern $321$ e che genera un pattern $231$ \'e $3\overline{5}241$ \\
\\Se invece la preimmagine deve contenere e generare un pattern $231$ si deve avere la presenza di un valore maggiore di $3$ tra $3$ e $1$, che faccia uscire $3$ dalla pila, ovvero il pattern $2341$.\\\\
Si uniscono adesso i due insiemi ottenuti:
$$S^{-1}(Av(231))=Av(2341,3\overline{5}241), S^{-1}(Av(321))=Av(34251, 35241, 45231)$$
Si osserva che $2341\preceq 34251$, quindi $34251$ non \'e minimo.\\
Inoltre i pattern $35241, 3\overline{5}241$, possono essere rappresentati dal pattern minimo $3241$ che rende non minimo anche $34251$.\\Il risultato che si ottiene \'e:$$(BS)^{-1}(Av(21))=Av(2341,3241,45231)$$
\section*{Combinazioni che terminano con \textit{queuesort}}
Nonostante anche per \textit{queuesort} sia stato trovato un algoritmo per le preimmagini\cite{magnusson2013sorting}, approcciare il problema "manualmente" analizzando i possibili comportamenti di \textit{queuesort} rispetto alle diverse possibili permutazioni risulta essere pi\'u semplice e pi\'u comprensibile.
\paragraph*{left-to-right-maxima}Nel valutare il comportamento di \textit{queuesort} \'e importante considerare la nozione di \textit{left-to-right maxima}, ovvero i valori in una sequenza che risultano maggiori di tutti i valori precedenti ad essi.\\Applicando \textit{queuesort} ad una sequenza tutti i \textit{left-to-rigth-maxima} vengono inseriti nella coda.
\subsection*{$S\circ{Q}$}$$(SQ)^{-1}(Av(21))=Q^{-1}S^{-1}(Av(21))=Q^{-1}(Av(231))$$Ancora una volta si considerano come candidati i pattern minimi che contengano almeno le coppie disordinate $(2,1),(3,1)$ ovvero $231,321$.\\\\
Se la preimmagine ha un pattern $321$ \'e necessario che l'elemento $3$ entri nella coda per generare un pattern $231$, questo si verifica se esso \'e un \textit{left-to-right-maxima}.\\$2$ viene sicuramente aggiunto all'output prima che $3$ esca dalla coda, poi un valore maggiore di $3$ deve essere presente prima di $1$, per aggiungere $3$ all'output prima che $1$ effettui il bypass.\\
L'unico pattern che rispetta tutte queste condizioni \'e $\overline{5}3241$.\\\\
Per ottenere un'immagine che contenga un pattern $231$ da una preimmagine che lo contiene a sua volta \'e necessario che $2$ effettui un bypass o che venga estratto dalla coda prima che $3$ vi entri. Il che implica la presenza di un valore $2^+>2$ prima di $2$ o tra $2$ e $3$.\\Similmente $3$ deve essere aggiunto all'output prima di $1$, qundi deve bypassare o essere estratto prima che $1$ bypassi. Il che implica l'esistenza di un elemento $4$ in posizione precdente a $1$; si noti che se $4$ compare prima di $2$ il pattern minimo \'e dato da $2^+=4$.\\
Gli unici pattern minimi che soddisfano queste condizioni sono $2431,4231$. Anche i pattern $42351,52341,25341,24351$ possono essere ottenuti dalle stesse condizioni, ma contengono tutti almeno uno dei due pattern elencati prima.
\\\\Si nota inoltre che, il pattern calcolato prima $\overline{5}3241\in Q^{-1}(Av(321))$ risulta adesso superfluo in quanto $4231\preceq\overline{5}3241$
$$(SQ)^{-1}(Av(21))=Av(2431,4231)$$
\subsection*{$B\circ{Q}$}$$(BQ)^{-1}(Av(21))=Q^{-1}B^{-1}(Av(21))=Q^{-1}(Av(231,321))$$
Ancora una volta, si osserva che $Q^{-1}(Av(231))$ \'e gi\'a stato calcolato nel caso precedente, e resta dunque da unire i risultati gi\'a ottenuti con $Q^{-1}(Av(321))$.\\
Si osserva che ogni candidato per la preimmagine del pattern $321$ deve a sua volta contenere un pattern $321$.\\
In particolare \'e necessario che tutti gli elemenri che compongono il pattern $321$ effettuino un bypass. L'unico pattern minimo che soddisfa queste condizioni \'e $4321$.\\Si combina quest'ultimo risultato e con quelli del caso precedente e si ottiene:$$(BQ)^{-1}(Av(21))=Av(2431,4231,4321)$$
\section*{Contenitori POP}
Lo studio di combinazioni di operatori che utilizzano contenitori POP si rivela pi\'u difficile, in quando si tratta di casi non approfonditi in letteratura.\\
Sono noti i pattern che rendono una permutazione non ordinabile, come presentato nel capitolo 2, ovvero:$$S_{POP}^{-1}(Av(21)) = Av(231,312)$$$$P_{POP}^{-1}(Av(21))=Av(132,321)$$
Si considerano due distinti casi: quello in cui l'operatore della combinazione applicato per primo alla permutazione \'e un algoritmo regolare e quello in cui \'e un algoritmo POP. Il primo caso \'e di pi\'u facile soluzione: infatti si tratta di cercare le preimmagini dell'algoritmo regolare dei pattern che rendono la permutazione non ordinabile dall'algoritmo POP, seguendo le metodologie applicate finora (diversi risultati gi\'a ottenuti per i casi regolari possono essere riutilizzati in questa circostanza).
\subsection*{$S_{POP}\circ{S}$}$$(S_{POP}\circ{S})^{-1}(Av(21))=S^{-1}(Av(231,312))$$
Si \'e gi\'a calcolato che $S^{-1}(Av(231))=Av(2341, 3\overline{5}241)$, quindi manca da calcolare le preimmagini di $312$.\\
I pattern candidati che possono generare $312$ sono quelli che contengono le coppie $(3,1),(3,2)$, ovvero $321, 312$.
Perch\'e un pattern $321$ generi un $312$ tramite \textit{stacksort} \'e necessario che $3$ entri in pila, ne esca prima che $2$ vi entri, e che quando $2$ vi entri non venga estratto prima di $1$. Deve quindi essere presente un valore $4$ posizionato tra $3$ e $2$ che provovhi l'uscita di $3$, e bisogna assicurare che non vi sia un valore $2^+>2$ tra $2$ e $1$. Queste condizioni sono realizzate da i pattern nella forma $342\overline{2^+}1$, in cui bisogna valutare le condizioni di ordine di $2^+$ rispetto agli elementi $3$ e $4$, differenziando tra i seguenti casi $(2^+<3), (3<2^+<4), (2^+>4)$ ed ottenendo i pattern $452\overline{3}1, 352\overline{4}2, 342\overline{5}1$.\\