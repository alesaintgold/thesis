\chapter{Composizioni di operatori di ordinamento}
Da questo momento si mostreranno le varie combinazioni di operatori e le relazioni tra relative classi di pattern e le loro controimmagini.
\paragraph*{controimmagini} Sia $X$ un operatore di ordinamento, la controimmagine di una certa permutazione $p$, secondo $X$, indicata con $X^{-1}(p)$ rappresenta l'insieme di tutte le permutazioni la cui immagine secondo l'operatore $X$ \`e uguale a $p$.$$X^{-1}(p) = \{\beta : p = X(\beta)\}$$\\
$Av(21)$ \`e l'insieme formato dalle sole permutazioni identit\'a, dato che contiene tutte le permutazioni in cui nessun elemento sia disordinato rispetto ad un altro, ovvero solo le permutazioni crescenti. Sia $X$ un operatore di ordinamento si indica con $X^{-1}(Av(21))$ l'insieme di tutte le permutazioni ordinabili da $X$.\\\\
Ad esempio si scrive $B^{-1}(Av(21)) = Av(231,321)$ per indicare che l'insieme di permutazioni ordinabili dall'operatore bubblesort coincide con l'insieme di permutazioni che evitano i pattern 231,321\cite{claesson2012sorting}.
\paragraph*{Composizione di operatori} Dati due operatori di ordinamento $X$ e $Y$, la loro composizione verr\'a indicata con $ XY = X \circ Y$, quindi, ad esempio, la composizione di \textit{Stacksort} e \textit{Bubblesort} (in questo ordine) si indica con $SB(\pi) = (S \circ B)(\pi) = S(B(\pi))$.\\\\
Siano $X,Y$ due operatori tali che sia noto l'insieme $X^{-1}Av(21) = Av(m)$, dove $m$ \`e un insieme minimo di pattern, allora l'insieme di permutazioni ordinabili dalla loro composizione $XY$ \`e dato da:
$$XY^{-1}(Av(21)) = Y^{-1}X^{-1}(Av(21)) = Y^{-1}Av(m)$$ 
Dunque conoscendo quali pattern debbano essere evitati per garantire l'ordinabilit\'a da parte di $X$ si possono ricercare le controimmagini di quegli stessi pattern secondo $Y$ per avere l'insieme ordinabile dalla loro combinazione.\\Ogni analisi delle composizioni mostrate in questa tesi inizier\'a con un espressione che mostri quindi quali controimmagini e secondo quale operatore \`e necessario cercare.
\paragraph*{Algoritmi per il calcolo di controimmagini} Sono gi\`a stati prodotti alcuni algoritmi per calcolare le controimmagini di pattern secondo gli operatori \textit{bubblesort}\cite{albert2010inverse}, \textit{stacksort}\cite{claesson2012sorting}, \textit{queuesort}\cite{magnusson2013sorting} e \textit{POP-stacksort}\cite{magnusson2013sorting}, che verranno esposti in seguito e utilizzati per l'analisi delle composizioni.
\paragraph*{Un caso gi\`a noto: Combinazione di Stacksort con Bubblesort} Si considera dunque la composizione $SB = S\circ{B}$ e ci si chiede quali permutazioni possano essere ordinate da essa.$$(SB)^{-1}Av(21)=B^{-1}S^{-1}(Av(21))=B^{-1}(Av(231))$$
Dunque ricercando per quali permutazioni \textit{bubblesort} evita il pattern 231 si trovano le condizioni per cui una permutazione risulta ordinabile da \textit{SB}.\\
Utilizzando l'algoritmo per le controimmagini di bubblesort\cite{albert2010inverse} si ottiene il seguente risultato, che verr\'a verificato in seguito:
$$(SB)^{-1}(Av(21))=Av(3241, 2341, 4231, 2431)$$
\section{Algoritmo per le controimmagini secondo bubblesort}
Per ottenere una controimmagine di bubblesort di un pattern classico, si applicano le seguenti regole:
\begin{description}
	\item si cercano tutte le inversioni nell'immagine in esame: poich\'e bubblesort non produce nuove inversioni, tutte le inversioni nell'immagine dovranno gi\`a essere presenti nella controimmagine;
	\item si considera una lista di \textbf{candidati}, composta da ogni pattern minimale che contiene almeno le stesse inversioni;
	\item per ogni candidato, si processa ogni sua inversione $(b,a)$ nel modo seguente:
	\begin{description}
	\item se $(b,a)$ \`e contenuta anche nell'immagine si ha che $b$ non \`e un LTR massimo o che lo \`e ma c'\'e un altro LTR massimo tra $b$ e $a$, quindi si decora la coppia come segue, producendo un nuovo pattern:
	\begin{center}
		\mmpattern{scale=1.4}{2}{1/2,2/1}{}{}$\Rightarrow$\mmpattern{scale=1.4}{2}{1/2,2/1}{}{0/2/2/3/1};
	\end{center}
	\item se $(b,a)$ \`e un'inversione del candidato che non \`e nell'immagine allora la coppia viene ordinata da una passata di bubblesort, si ha che $b$ \`e l'ultimo LTR massimo prima di $a$, la coppia viene decorata nel modo seguente:
	\begin{center}
		\mmpattern{scale=1.4}{2}{1/2,2/1}{}{}$\Rightarrow$\mmpattern{scale=1.4}{2}{1/2,2/1}{0/2,1/2}{};\\
	\end{center}
	\end{description}
	\item applicare ad un candidato le condizioni per tutte le sue inversioni (considerando che se una zona \`e oscurata da una condizione, non pu\`o essere decorata da un altra) fornisce un mesh-pattern decorato che descrive il pattern contenuto nella controimmagine.
\end{description}
\section{Combinazioni di un operatore di ordinamento con bubblesort}
\subsection{Esempio pratico di calcolo di una controimmagine: combinazione di stacksort con bubblesort}
Prima di tutto si ricerca di quale classe di pattern nonch\'e tramite quale operatore occorre calcolare le controimmagini:
$$(SB)^{-1}(Av(21)) = B^{-1}(S^{-1}(Av(21)))=B^{-1}(Av(231))$$
Si ricorda che vale che $S^{-1}Av(21) = Av(231)$
I pattern candidati ad essere controimmagini di 231 sono i pattern minimale che contengono le coppie $(2,1),(3,1)$, ovvero i pattern 231,321. Si esaminano separatamente i due candidati.
\paragraph*{231:} si osserva che una condizione per l'inversione $(2,1)$ \`e gi\`a realizzata dall'elemento $3$, applicando le regole descritte prima all'inversione $(3,1)$ si ha il mesh-pattern \mmpattern{scale=0.7}{3}{1/2,2/3,3/1}{}{0/3/3/4/1}, evitare quest'ultimo equivale ad evitare contemporaneamente i pattern classici 2341, 2431, 4231.
\paragraph*{321:} in questo caso si ha l'inversione $(3,2)$ che deve essere invertita, quindi si inseriscono le aree oscurate, in seguito si osserva che, come prima, $3$ realizza la condizione per l'inversione $(2,1)$ e dunque basta aggiungere le decorazioni per l'inversione $(3,1)$ tenendo conto delle aree oscurate. Di seguito si mostrano i passaggi:
\begin{center}
\mmpattern{scale=1.5}{3}{1/3,2/2,3/1}{}{}$\Rightarrow$
\mmpattern{scale=1.5}{3}{1/3,2/2,3/1}{0/3,1/3}{}$\Rightarrow$
\mmpattern{scale=1.5}{3}{1/3,2/2,3/1}{0/3,1/3}{2/3/3/4/1}$\Rightarrow$
\mmpattern{scale=1.3}{4}{1/3,2/2,3/4,4/1}{0/3,1/3,0/4,1/4}{}
\end{center}
Il pattern finale \`e simile ad un pattern classico 3241, presentando in pi\'u delle aree oscurate. Si pu\`o tuttavia osservare come un eventuale elemento contenuto in quelle aree oscurare pu\`o solo formare un ulteriore pattern 3241 (assumendo il ruolo del 3) o un pattern 4231 con gli ultimi 3 elementi del pattern.\\
In ultima analisi dunque evitare il mesh-pattern trovato, tenendo anche conto degli altri pattern trovati, equivale ad evitare il pattern classico 3241 che, unito agli altri fornisce esattamente il risultato atteso $(SB)^{-1}(Av(21))=Av(2341, 2431, 4231, 3241)$.\\\\
Si osservi che il software \texttt{permutasort}, presentato nel capitolo 3, pu\`o aiutare ad arrivare alle stesse conclusioni. Infatti lanciando il programma sull'operatore SB e con $n=4$ si ottengono i seguenti risultati:\\\\\texttt{\$ python permutasort.py 4 SB }\\\dots\\\texttt{The following 4 4-permutations are not sortable with the operator SB:}\\\texttt{(2, 3, 4, 1)}\\\texttt{(2, 4, 3, 1)}\\\texttt{(3, 2, 4, 1)}\\\texttt{(4, 2, 3, 1)}\\\dots\\\\
Questo risultato \`e utile come indicazione, ma non fornisce garanzia di aver trovato i pattern corretti: infatti non si pu\`o escludere la presenza di pattern classici pi\'u lunghi n\'e la presenza di pattern barrati.\\
Per quanto quindi questo strumento si sia rivelato molto utile per avere un'idea di quali pattern aspettarsi prima di ricercarli, soprattutto durante le combinazioni pi\'u complesse che presenteremo in seguito, non pu\`o sostituire del tutto un approccio pi\'u formale e teorico, come l'applicazione di uno degli algoritmi introdotti prima.
\subsection{Combinazione di queuesort con bubblesort}
$$(QB)^{-1}(Av(21))=B^{-1}Q^{-1}(Av(21))=B^{-1}(Av(321))$$
Si osserva che le inversioni in $321$ sono $(3,2),(3,1),(2,1)$ e l'unico pattern minimo che le contiene tutte \`e appunto $321$, che quindi \`e l'unico candidato.\\
In questo caso a tutte le inversioni deve essere applicata la prima condizione, a parte a $(2,1)$ perch\'e la condizione \`e gi\`a soddisfatta da 3. Il pattern decorato che si ottiene \`e:
\begin{center}
\mmpattern{scale=1.5}{3}{1/3,2/2,3/1}{}{0/3/3/4/{\dots\dots1},0/3/2/4/1},
\end{center}
ovvero  l'unione di \mmpattern{scale=0.7}{3}{1/3,2/2,3/1}{}{0/3/2/4/1} e \mmpattern{scale=0.7}{3}{1/3,2/2,3/1}{}{0/3/3/4/1}. Tuttavia questi due pattern possono essere semplificati nel primo: si nota infatti che \mmpattern{scale=0.7}{3}{1/3,2/2,3/1}{}{0/3/2/4/1}, che si ottiene dalle condizioni per $(3,2)$ soddisfa anche le condizioni per $(3,1)$ e inoltre vale che \mmpattern{scale=0.7}{3}{1/3,2/2,3/1}{}{0/3/2/4/1}$\preceq$\mmpattern{scale=0.7}{3}{1/3,2/2,3/1}{}{0/3/2/4/1,2/3/3/4/1}.\\
Quindi le controimmagini del pattern 321 tramite bubblesort sono date da:
\begin{center}
\mmpattern{scale=1.4}{3}{1/3,2/2,3/1}{}{0/3/2/4/1}.
\end{center}
Questo mesh pattern corrisponde ai pattern classici $4321,3421$.$$QB^{-1}(Av(21))=Av(4321,3421)$$
\section{Composizioni di un operatore di ordinamento con stacksort}
L'algoritmo per le controimmagini di \textit{stacksort} \`e abbastanza simile a quello per \textit{bubblesort}.\\
Si ricava una lista di candidati in modo equivalente al primo algoritmo, e ogni candidato viene processato applicando le seguenti regole ad ogni inversione $(b,a)$ presente nel candidato:
\begin{description}
\item se $(b,a)$ \`e presente anche nell'immagine allora deve essere presente un elemento $c>b$ tra $b$ e $a$ che fa uscire $b$ dalla pila prima che $a$ vi entri
\begin{center}
\mmpattern{scale=1}{2}{1/2,2/1}{}{} $\Longrightarrow$\mmpattern{scale=1}{2}{1/2,2/1}{}{1/2/2/3/1} 
\end{center}
\item se invece $(b,a)$ non \`e presente nell'immagine allora si deve escludere la presenza di tale elemento
\begin{center}
\mmpattern{scale=1}{2}{1/2,2/1}{}{} $\Longrightarrow$\mmpattern{scale=1}{2}{1/2,2/1}{1/2}{} 
\end{center}
\end{description}
\subsection{Composizione  di {queuesort} con {stacksort}}
$$(QS)^{-1}(Av(21))=S^{-1}Q^{-1}(Av(21))=S^{-1}(Av(321))$$
Si ricercano dunque le controimmagini di $321$ secondo \textit{stacksort}.\\$321$ \`e l'unico candidato, quindi sappiamo che tutte le controimmagini devono contenere il pattern $321$.\\
L'unione delle condizioni per le inversioni $(3,2),(2,1)$ minimizza anche la condizione per l'inversione $(3,1)$, dunque il pattern che si ottiene \`e il seguente:
\begin{center}
\mmpattern{scale=1.5}{3}{1/3,2/2,3/1}{}{1/3/2/4/1,2/2/3/4/1}.
\end{center}
Per cercare i pattern classici equivalenti si introducono i seguenti elementi:
\begin{description}
	\item[$3^+>3$] rappresenta l'elemento da inserire nella decorazione di sinistra
	\item[$2^+>2$] rappresenta l'elemento da inserire nella decorazione di destra
\end{description}
Dunque i pattern classici corrispondenti al mesh-pattern trovato devono essere nella forma $33^+22^+1$.\\
Se $2^+<3$ allora la controimmagine assume la forma del pattern $45231$.\\
Altrimenti se $2^+>3$, si possono ottenere due diverse controimmagini:
\begin{description}
	\item[$3^+>2^+$] genera il pattern $35241$
	\item[$2^+>3^+$] genera il pattern $34251$
\end{description}
$$(QS)^{-1}(Av(21)) = Av(34251, 35241, 45231)$$
\subsection{Composizione di {bubblesort} con {stacksort}}
$$(BS)^{-1}(Av(21))=S^{-1}B^{-1}(Av(21))=S^{-1}(Av(231,321))$$
Gi\`a dall'analisi della combinazione precedente \`e risultato che $S^{-1}(Av(321))=Av(34251, 35241, 45231)$ quindi \`e necessario calcolare solo $S^{-1}(Av(231))$. Quest'ultimo risultato \`e stato ampiamente studiato in letteratura, in quanto analogo al caso di una variante di \textit{stacksort} che utilizza 2 pile. Il risultato che si ottiene \`e dunque che $S^{-1}(Av(231))=Av(2341, 3\overline{5}241)$\cite{claesson2012sorting}.\\
Si uniscono dunque i due risultati:
$$S^{-1}(Av(231))=Av(2341,3\overline{5}241), S^{-1}(Av(321))=Av(34251, 35241, 45231)$$
Si osserva che $2341\preceq 34251$, quindi $34251$ non \`e minimo.\\
Inoltre i pattern $35241, 3\overline{5}241$ insieme, possono essere semplificati dal pattern $3241$, dato che questo debba essere evitato sia quando \`e presente un elemento 5 tra 3 e 2, sia quando \`e assente. Questo ci permette di semplificare anche $34251$, dato che contiene un'occorrenza del pattern $3241$ nella sottosequenza 3251\\
Il risultato che si ottiene \`e:
$$(BS)^{-1}(Av(21))=Av(2341,3241,45231)$$
\section{Combinazioni di un operatore di ordinamento con \textit{queuesort}}
Come per i due casi precedenti esiste un algoritmo per trovare controimmagini di un pattern applicando decorazioni al relativo mesh-pattern.\\
Come per i casi precedenti si stabilisce una lista di pattern candidati, e li si esaminano ad uno ad uno, applicando le seguenti condizioni ad ogni inversione $(b,a)$:
\begin{description} 
	\item Se $(b,a)$ \`e presente anche nell'immagine \`e perch\'e non viene ordinata da una passata di queuesort, questo si verifica sicuramente quando $b$ non \`e un LTR massimo ($a$ non lo \`e mai, in quanto minore di $b$), se invece $b$ risulta essere un LTR massimo allora deve essere presente un ulteriore inversione $(d,c)$ tra $b$ e $a$, cos\'i che $d$ si accodi e $c$ provochi l'estrazione di $b$ prima che $a$ possa effettuare un bypass. Queste condizioni si applicano decorando l'inversione in uno dei modi seguenti
	\begin{center}
		\decpatternww{scale=1.7}{2}{1/2,2/1}{}{}{}{}{} $\Rightarrow$
		\decpatternww{scale=1.7}{2}{1/2,2/1}{}{}{}{0/2/1/3/1}{},
		\decpatternww{scale=1.7}{2}{1/2,2/1}{}{}{0/2}{1/2/2/3/{\mpattern{scale=0.5}{2}{2/1,1/2}{}}}{}
	\end{center}
	\item Se un'inversione contenuta nel candidato non \`e presente nell'immagine significa che nessuna delle condizioni espresse prima viene soddisfatta, la decorazione che si applica rappresenta una negazione delle condizioni mostrate prima
	\begin{center}
		\decpatternww{scale=1.7}{2}{1/2,2/1}{}{}{}{}{} $\Rightarrow$
		\decpatternww{scale=1.7}{2}{1/2,2/1}{}{}{0/2}{}{1/2/2/3/{\mpattern{scale=0.5}{2}{2/1,1/2}{}}}
	\end{center}
\end{description}
\subsection{Composizione di {stacksort} con {queuesort}}
$$SQ^{-1}(Av(21)) = Q^{-1}S^{-1}(Av(21)) = Q^{-1}(Av(231))$$
Si ricercano dunque le controimmagini di 231 secondo queuesort. Gli stessi elementi che formano un pattern 231 nell'immagine possono formare nella controimmagine lo stesso pattern 231 o un pattern 321. Si esaminano i due casi separatamente.\\\\
Si applicano alle inversioni di 231 le possibili condizioni espresse precedentemente, ottenendo i seguenti mesh-pattern:
\begin{center}
\textbf{(A)}\decpatternww{scale=1.5}{3}{1/2,2/3,3/1}{}{}{}{0/3/1/4/1}{},
\textbf{(B)}\decpatternww{scale=1.5}{3}{1/2,3/1}{}{}{0/2,0/3}{1/2/3/4/{\mpattern{scale=0.8}{2}{2/1,1/2}{}}}{},
\textbf{(C)}\decpatternww{scale=1.5}{3}{1/2,2/3,3/1}{}{}{0/3,1/3}{0/2/1/4/1,2/3/3/4/{\mpattern{scale=0.5}{2}{2/1,1/2}{}}}{},
\textbf{(D)}\decpatternww{scale=1.5}{3}{1/2,2/3,3/1}{}{}{0/3,1/3}{2/3/3/4/{\mpattern{scale=0.5}{2}{2/1,1/2}{}}}{}
\end{center}
Si osserva che i pattern (D) e (C) contengono un'occorrenza del pattern (B) e dunque evitare (B) comprende evitare anche (C) e (D).\\
Il pattern (A) corrisponde al pattern classico 4231, mentre il pattern (B) \`e simile ad un pattern 2431 con alcune aree oscurate.\\\\
Esaminiamo questo caso in cui una permutazione eviti il pattern (B) pi\'u approfonditamente: una permutazione evita questo pattern se in ogni occorrenza di un pattern 2431 l'elemento corrispondente al 2 del pattern non \`e un LTR massimo. Si prenda il primo LTR massimo alla sinistra dell'elemento 2 (tale elemento esiste dato che stiamo esaminando il caso  di una permutazione che eviti (B)): questo entra in coda, 2 viene aggiunto all'output, e almeno uno tra l'elemento precedente e gli elementi della coppia $43$ deve essere aggiunto all'output; ognuno di questi elementi \`e maggiore di 2 e dunque si va a formare un pattern 231 quando 1 bypassa. Se ne deduce che, ai fini di evitare la produzione di pattern 231, il fatto che una permutazione eviti il pattern (B) implica che essa eviti anche il pattern 2431, dato che la presenza di elementi nelle aree oscurate non previene la formazione di pattern 231.\\\\
Si cercano adesso le controimmagini di 231 che contengono 321, applicando le regole dell'algoritmo alle inversioni di 321.\\
Applicando le regole esposte, ed escludendo quelle che si contraddicono, si ottiene il seguente mesh-pattern: 
\begin{center}
\decpatternww{scale=1.7}{3}{1/2,2/3,3/1}{}{}{0/3}{2/3/3/4/{\mpattern{scale=0.5}{2}{2/1,1/2}{}}}{1/3/2/4/{\mpattern{scale=0.5}{2}{2/1,1/2}{}}}
\end{center}
che contiene un'occorrenza del pattern 2431 trovato prima, quindi anche questo pu\`o essere semplificato.\\\\
Si ottiene che:$$SQ^{-1}(Av(21)) = Av(2431, 4231)$$
\subsection{Composizione di {bubblesort} con {queuesort}}
$$BQ^{-1}(Av(21)) = Q^{-1}(B^{-1}(Av(21))) = Q^{-1}(Av(231,321))$$
L'insieme $Q^{-1}Av(231)$ \`e gi\`a stato calcolato nella sezione precedente, quindi resta da cercare l'insieme $B^{-1}Av(321)$ ed eseguire l'intersezione tra i due.\\321 \`e l'unico candidato per la ricerca delle controimmagini, ed inoltre l'elemento 3 soddisfa gi\`a una delle condizioni per l'inversione $(2,1)$. Applicando le condizioni alle inversioni $(3,1),(3,2)$, si ottengono i seguenti mesh-pattern:
\begin{center}
\decpatternww{scale=1.5}{3}{1/3,2/2,3/1}{}{}{}{0/3/1/4/1}{},
\decpatternww{scale=1.5}{3}{1/3,2/2,3/1}{}{}{}{1/3/2/4/{\mpattern{scale=0.5}{2}{2/1,1/2}{}}}{}.
\end{center}  
Entrambi corrispondono a pattern classici, rispettivamente 4321, 35421. Si noti che il secondo contiene un'occorrenza del primo, che quindi \`e l'unico da considerare. Si unisce questo risultato a quelli del capitolo precedente.
$$BQ^{-1}(Av(21)) = Av(2431, 4231, 4321)$$ 
\section{Contenitori POP}
%frase da riscrivere
Lo studio di combinazioni di operatori che utilizzano contenitori POP si rivela pi\'u difficile, in quando si tratta di casi non approfonditi in letteratura.\\
Sono tuttavia noti i pattern che rendono una permutazione non ordinabile, come presentato nel capitolo 2, ovvero:
$$S_{POP}^{-1}(Av(21)) = Av(231,312)$$$$Q_{POP}^{-1}(Av(21))=Av(321,2413)$$
Grazie a questi dati, \`e possibile individuare un caso di facile risoluzione: ovvero quello in cui un algoritmo che usa contenitori POP \`e concatenato ad un operatore regolare.
$$(X_{POP}\circ{Y})(\pi)= X_{POP}(Y(\pi)))\Rightarrow(X_{POP}\circ{Y})^{-1}(\pi) = Y^{-1}X_{POP}^{-1}(\pi)=Y^{-1}(Av(m))$$
Sapendo l'insieme di pattern $m$ che rendono la permutazione non ordinabile dall'operatore POP baster\'a cercare le loro controimmagini secondo l'operatore regolare con i metodi usati finora, come verr\'a mostrato nelle sezioni seguenti.
\subsection{POP Queuesort}L'algoritmo POP queuesort richiede una diversa analisi da POP stacksort.\\
Esistono diverse versioni di \textit{POP queuesort} ed in particolare ne esistono due ottimali\cite{cioni2021sorting}: \textit{Min} e \textit{Cons}.
\begin{description}
	\item[\textit{Min}] in questa versione l'operazione di POP viene eseguita solo se il primo elemento della coda \`e il successivo dell'ultimo elemento aggiunto all'output; se l'elemento in input \`e maggiore dell'ultimo elemento della coda (o se la coda \`e vuota) allora viene accodato mentre negli altri casi, se l'elemento dell'input \`e minore della testa della coda allora bypassa altrimenti si esegue un POP.
	\item[\textit{Cons}] questa versione si basa sull'idea di avere sempre elementi consecutivi nella coda; \`e la versione che verr\'a adottata in questa tesi; da qui in avanti ogni riferimento a \textit{POP-queuesort} sar\'a riferito a \textit{Cons}
\end{description}
Nell'algoritmo \textit{Cons} si definisce una coda POP come un insieme $Q$, inizialmente vuoto, di interi su cui \`e possibile fare le seguenti operazioni:
\begin{description}
	\item[Enqueue($\pi_i$, $Q$):] inserisce l'$i$-esimo elemento della permutazione $\pi$ in $Q$, accodandolo alla coda;
	\item[Pop($Q$):] estrae gli elementi contenuti in $Q$ e li aggiunge all'output nello stesso ordine a cui sono stati aggiunti a $Q$;
	\item[Bypass($\pi_i$):] posiziona l'$i$-esimo elemento di $\pi$ nell'output.
\end{description}
Inoltre si utilizzano le seguenti funzioni:
\begin{description}
	\item[Back($Q$):] restituisce il valore dell'ultimo elemento aggiunto a $Q$
	\item[Front($Q$):] restituisce il valore dell'elemento in $Q$ che vi \`e stato inserito per primo
\end{description}
\begin{algorithm}[H]
   \caption{Cons - POP Queuesort}
\begin{algorithmic}[1]
\State $Q\leftarrow\emptyset$ 
\For{$i=1$ {\bfseries to} $n$}
	\If{$Q=\emptyset$ \textbf{or} $\pi_i=Back(Q)+1$}
		\State Enqueue($\pi_i$)
	\Else
		\If{ $Front(Q)>\pi_i$}
			\State $Bypass(\pi_i)$
		\Else
			\State $Pop(Q)$
			\State $Enqueue(\pi_i)$
		\EndIf
	\EndIf
\EndFor
\If{$Q\neq\emptyset$}
\State $Pop(Q)$
\EndIf
\end{algorithmic}
\end{algorithm}
La dimostrazione che \textit{Cons} (cos\'i come \textit{Min}) sia un algoritmo ottimale nella classe degli algoritmi \textit{POP-queuesort} si ha dal fatto che esso ordina tutte e sole le sequenze dell'insieme Av(321, 2413), che sono esattamente tutte e sole le permutazioni ordinabili con una POP queue\cite{cioni2021sorting}.
\section{Composizioni che includono operatori POP}
\subsection{Composizione di {POP-stacksort} con {stacksort}}
$$(S_{POP}\circ{S})^{-1}(Av(21))=S^{-1}(Av(231,312))$$
Si \`e gi\`a calcolato che $S^{-1}(Av(231))=Av(2341, 3\overline{5}241)$, quindi manca da calcolare le controimmagini di $312$.\\
I pattern candidati che possono generare $312$ sono quelli che contengono le coppie $(3,1),(3,2)$, ovvero $321, 312$.
Perch\'e gli elementi di un pattern $321$ nella controimmagine generino un pattern $312$ tramite \textit{stacksort} \`e necessario che $3$ entri in pila e ne esca prima che $2$ vi entri, deve quindi essere presente un valore $4$ posizionato tra $3$ e $2$ che provochi l'uscita di $3$, ovvero il pattern $3421$.\\
Se gli elementi del pattern $312$ nell'immagine formano lo stesso pattern anche nella controimmagine allora anche in questo caso bisogna assicurarsi che $3$ venga estratto dalla pila prima di $1$, si ha quindi il pattern $3412$.\\Si uniscono tutti i risultati:
$$(S_{POP}S)^{-1}(Av(21))=Av(2341, 3412, 3421, 3\overline{5}241)$$
\subsection{Composizione di {POP-stacksort} con {bubblesort}}
$$(S_{POP}\circ{B})^{-1}(Av(21))=B^{-1}(Av(231,312))$$
Anche in questo caso la controimmagine di $231$ \`e stata calcolata: $B^{-1}(Av(231)) = (S\circ{B})^{-1}(Av(21)) = Av(2341, 2431, 3241, 4231)$.\\
Si ricerca dunque le controimmaigini di 312 secondo bubblesort. \\
Gli elementi del pattern $312$ nell'immagine possono solo formare un pattern $312$ o $321$ nella controimmagine.\\
Si pu\`o osservare come gli stessi elementi di un pattern $321$ non possono divenire un $312$. La presenza dell'elemento maggiore del pattern all'inizio di essa evita l'ordinamento degli elementi successivi.\\
Perch\'e un pattern $312$ rimanga invariato \`e sufficiente la presenza di un elemento maggiore di tutto il pattern in posizione tale da evitare l'ordinamento di $3$ con qualsiasi elemento, applicando le regole descritte prima per \textit{bubblesort}: 
\begin{center}
\mmpattern{scale=1.5}{3}{1/3,2/1,3/2}{}{0/3/2/4/1}$=4312,3412$
\end{center}
$$(S_{POP}{B})^{-1}(Av(21))=Av(2341, 2431, 3241, 3412, 4231, 4312)$$
\subsection{Composizione di {POP-stacksort} con {queuesort}}
$$(S_{POP}\circ{Q})^{-1}(Av(21))=Q^{-1}(Av(231,312))$$
\`E gi\`a noto dai calcoli nelle sezioni precedenti che $Q^{-1}(Av(231)=Av(4231,2431)$. Si procede dunque a ricercare controimmagini di 312.\\
I candidati per la produzione di controimmagini di $312$ sono $312, 321$. Applicando le regole dell'algoritmo per le controimmagini secondo queuesort alle inversioni $(3,1),(3,2)$ si ottengono i seguenti mesh-pattern:
\begin{center}
\decpatternww{scale=1.5}{3}{1/3,2/1,3/2}{}{}{}{0/3/1/4/1}{},
\decpatternww{scale=1.5}{3}{1/3,2/1,3/2}{}{}{0/3}{1/3/2/4/{\mpattern{scale=0.5}{2}{2/1,1/2}{}}}{}.
\end{center}
Il primo dei due mesh pattern corrisponde al pattern classico 4312, che \`e contenuto anche nel secondo.\\
Usando invece 321 come candidato si osserva come l'inversione $(2,1)$ richieda di essere invertita, applicando la relativa condizione si ottiene di aggiungere un area oscurata che include la posizione dell'elemento 3. Dunque il pattern stesso contraddice le condzioni dell'algoritmo e non \`e possibile trovare una controimmagine partendo da questo candidato.\\
$$(S_{POP}{Q})^{-1}(Av(21))=Av(2431,4231,4312)$$
\subsection{Composizione di {POP-queuesort} con {stacksort}}
$$Q_{POP}S^{-1}(Av(21)) = S^{-1}Q_{POP}^{-1}(Av(21)) = S^{-1}(Av(321, 2413))$$
Si osserva che una parte dei risultati richiesti \`e gi\`a stata calcolata: $S^{-1}(Av(321)) = Av(34251, 35241, 45231)$.\\
Si osserva come questi pattern hanno una forma in comune: essi sono infatti tutti composti da una non inversione seguiti da un pattern 231, dove gli elementi $2,1$ di quest'ultimo sono minori della non inversione iniziale. Di seguito si illustra questa forma con un pattern decorato che chiamiamo (A).
\begin{center}
\textbf{(A)}\mmpattern{scale=1.5}{4}{1/3,2/4,3/2,4/1}{}{3/2/4/5/1}
\end{center}
In questo caso risulta utile evidenziare la forma di questo specifico pattern in quanto ricorrente anche in quelli che troveremo successivamente, e quindi render\`a pi\'u semplice individuare le occorrenze di questi pattern gi\`a noti in quelli che verranno trovati per semplificarli.\\\\
Si ricerca le controimmagini del pattern 2413 secondo stacksort. I candidati sono: 2413, 2431, 4213, 4231, 4321.
\paragraph*{2413:} Le inversioni di 2413 sono $(2,1),(4,1),(4,3)$. Si osserva che 4 soddisfa gi\`a la condizione per l'inversione $(2,1)$, e decorando il pattern per soddisfare la condizione per $(4,1)$ si ottiene
\begin{center}\mmpattern{scale=1.5}{4}{1/2,2/4,3/1,4/3}{}{2/4/3/5/1}\end{center} che soddisfa anche la condizione per l'inversione $(4,3)$, dato che il mesh pattern che si \`e cosi\'i ottenuto equivale al pattern classico 24513, quest'ultimo \`e l'unica controimmagine ricavato da questo candidato.
\paragraph*{2431:} Le inversioni di 2431 sono $(2,1),(4,3),(4,1),(3,1)$. La condizione per $(2,1)$ \`e gi\`a soddisfatta, applicando le ulteriori condizioni si ottiene il seguente mesh-pattern
\begin{center}
\mmpattern{scale=1.5}{4}{1/2,2/4,3/3,4/1}{3/3,3/4}{2/4/3/5/1},
\end{center}
che equivale al pattern classico 24531 con alcune aree barrate. Nell'esaminare le aree barrate si pu\`o osservare che se un elemento le occupasse questo formerebbe un occorrenza del pattern (A) con gli altri elementi del pattern a parte il 2. Se ne deduce che le aree barrate non determinano l'ordinabilit\`a o meno di una permutazione, quindi al fine di individuare le permutazioni ordinabili da $Q_{POP}S$, tenuto conto degli altri pattern individuati, queste possono essere trascurate e il mesh-pattern pu\`o essere considerato equivalente al pattern classico 24531. Simili osservazione verranno utilizzate anche per i successivi candidati. 
\paragraph*{4213:}Le inversioni di 4213 sono $(4,2),(4,1),(4,3),(2,1)$. Di queste $4,2$ non \`e presente in 2413, e quindi provoca l'inserimento di un area oscurata, poi per soddisfare la condizione per $(4,1)$ si procede ad aggiungere una decorazione che soddisfa anche le condizioni per le inversioni $(4,3)$ e $(2,1)$.
\begin{center}
\mmpattern{scale=1.5}{4}{1/4,2/2,3/1,4/3}{1/4}{2/4/3/5/1}
\end{center}
L'area oscurata pu\`o essere trascurata dato che la sua occupazione provoca un'occorrenza del pattern (A), e duqnue si ottiene il pattern classico 42513.
\paragraph*{4231:} Le inversioni di 4231 sono $(4,2),(4,3),(4,1),(2,1),(3,1)$. Le condizioni delle inversioni $(4,2)$ e $(3,1)$ richiedono di inserire delle aree oscurate mentre la decorazione aggiunta per l'inversione $(4,3)$ soddisfa anche la condizione per $(4,1),(2,1)$.
\begin{center}
\mmpattern{scale=1.5}{4}{1/4,2/2,3/3,4/1}{1/4,3/3,3/4}{2/4/3/5/1}
\end{center}
In questo, come nei casi precedenti, le aree oscurate si possono trascurare dato che la loro occupazione causa un'occorrenza del pattern (A).
\paragraph*{4321:} Le inversioni del pattern 4321 sono $(4,3),(4,2),(4,1),(3,2),(3,1),(2,1)$. Applicando tutte le regole si nota come queste siano contradditorie tra di loro, dunque questo candidato non pu\`o generare controimmagini.
\\\\Si uniscono tutti i risultati ottenuti:
$$Q_{POP}S^{-1}(Av(21)) = Av(24513, 24531, 34251, 35241, 42513, 45231)$$
\subsection{Composizione di {POP-queuesort} con {bubblesort}}
$$(Q_{POP}{B})^{-1}(Av(21)) = B^{-1}Q_{POP}^{-1}(Av(21))=B^{-1}(Av(321,2413))$$
La classe di pattern $B^{-1}(Av(321))$ \`e gi\`a stata calcolata:
$$B^{-1}(Av(321)) = QB^{-1}(Av(21)) = Av(3421, 4321)$$
Si ricerca adesso la classe di pattern corrispondente a $B^{-1}(Av(2413))$. La lista dei candidati \`e la seguente: 2413, 2431, 4213, 4231, 4321.
\paragraph*{Osservazione:} i pattern 2431, 4231, 4321 contengono l'inversione $(3,1)$ che non presente nel pattern dell'immagine. Questa specifica inversione \`e in posizione tale che nessuno dei due elementi sia un LTR massimo, e dunque l'inversione non possa essere ordinata. In questi 3 pattern infatti, l'elemento 4 si trova all'interno dell'area oscurata che si dovrebbe aggiungere durante l'applicazione dell'algortimo per le controimmagini di bubblesort all'inversione $(3,1)$. Ne consegue che i pattern  2431, 4231, 4321, pur essendo nella lista di candidati non possono generare il pattern 2413.\\
Si ricercano ora le controimmagini date dagli altri candidati:
\paragraph*{2413:}Si osserva che 4 soddisfa gi\`a la condizione per l'inversione $(2,1)$, dunque si prende l'intersezione tra le condizioni per le inversioni $(4,3),(4,1)$ ottenendo il seguente pattern decorato:
\begin{center}
\mmpattern{scale=1.2}{4}{1/2,2/4,3/1,4/3}{}{0/4/3/5/1}
\end{center}
Da questo si ottengono i pattern classici 52413, 25413, 24513.
\paragraph*{4213:} L'inversione $(4,2)$ deve essere ordinata, in quanto non presente in 2413.
\begin{center}
\mmpattern{scale=1.2}{4}{1/4,2/2,3/1,4/3}{}{} $\Rightarrow$
\mmpattern{scale=1.2}{4}{1/4,2/2,3/1,4/3}{0/4,1/4}{}
\end{center}
Si osserva poi che 4 soddsfa gi\`a le condizioni per l'inversione $(2,1)$, mentre applicando le condizioni per le altre inversioni, tenendo conto dell'area oscurata si ottiene:
\begin{center}
\mmpattern{scale=1.2}{4}{1/4,2/2,3/1,4/3}{0/4,1/4}{2/4/4/5/1}
\end{center}
Trovare l'insieme di pattern classici equivalente a quest'ultimo mesh-pattern non \`e immediato come lo \`e in altri casi, dunque si continuer\`a ad adottare la notazione dei mesh-pattern, invece di quella dei pattern classici, per rappresentare questa soluzione.\\\\Si uniscono i risultati ottenuti.
$$(Q_{POP}{B})^{-1}(Av(21)) = Av(3421, 4321, 24513, 25413, 52413, \mmpattern{scale=0.9}{4}{1/4,2/2,3/1,4/3}{0/4,1/4}{2/4/4/5/1})$$
\subsection{Composizione di {POP-queuesort} con {queuesort}}
$$Q_{POP}Q^{-1}Av(21) = Q^{-1}Q_{POP}(Av(21)) = Q^{-1}(Av(321, 2413))$$
Si ricercano le controimmagini di 321. Si osserva che 3 soddisfa gi\`a una delle condizione per l'inversione $(2,1)$. Applicando le condizioni alle altre inversioni, i pattern minimi che si ottengono sono i seguenti:
\begin{center}
\mmpattern{scale=1.7}{3}{1/3,2/2,3/1}{}{0/3/1/4/1}, 
\decpatternww{scale=1.7}{3}{1/3,2/2,3/1}{}{}{0/3}{1/3/2/4/{\mpattern{scale=0.5}{2}{2/1,1/2}{}}}{}
\end{center}
Il primo corrisponde al pattern classico 4321 che, essendo contenuto anche nel secondo, risulta essere l'unica controimmagine individuata.\\\\
Si ricercano adesso le controimmagini di 2413. Le inversioni di 2413 sono $(2,1),(4,1),(4,3)$, che ammettono i candidati 2413, 2431, 4213, 4231, 4321. Di questi 4321 \`e gi\`a stato trovato prima, dunque non vi verr\`a applicato l'agloritmo dato che ogni sua controimmagine ne contiene sicuramente un'occorrenza.
\paragraph*{2413:} I pattern minimi che si ottengono applicando le condizioni sono i seguenti:
\begin{center}
\mmpattern{scale=1.7}{4}{1/2,2/4,3/1,4/3}{}{0/4/1/5/1}, 
\decpatternww{scale=1.7}{4}{1/2,3/1,4/3}{}{}{}{1/3/3/5/{\mpattern{scale=0.5}{2}{2/1,1/2}{}}}{}.
\end{center}
Le condizioni relative a tutte le inversioni del candidato producono pi\`u di due pattern, ma per ragioni di brevit\`a si mostrano solo questi, di cui tutti gli altri contengono occorrenze, omettendo le semplificazioni. I due mesh-pattern generano rispettivamente i pattern classici 52413 e 25413.
\paragraph*{2431:} In questo candidato si osserva come l'inversione $(4,3)$ contraddica la condizione per l'inversione $(2,1)$, inoltre l'elemento 4 contraddice la condizione dell'inversione $(3,1)$, dunque questo candidato non pu\`o produrre controimmagini.
\paragraph*{4213:} Applicando le condizioni si ottiene il seguente mesh-pattern
\begin{center}
\decpatternww{scale=1.7}{4}{1/4,2/2,3/1,4/3}{}{}{0/4}{1/4/3/5/{\mpattern{scale=0.5}{2}{2/1,1/2}{}}}{1/4/2/5/{\mpattern{scale=0.5}{2}{2/1,1/2}{}}}
\end{center}
Le condizioni richiedono la presenza di un'inversione di elementi maggiori di 4 tra 4 e 1 e l'assenza di un'inversione simile tra 4 e 2. Questo pu\`o generare i seguenti mesh-pattern:
\begin{center}
\mpattern{scale=1.2}{6}{1/4,2/6,3/2,4/5,5/1,6/3}{0/4,0/5,0/5,0/6},
\mpattern{scale=1.2}{6}{1/4,2/2,3/6,4/5,5/1,6/3}{0/4,0/5,0/5,0/6}.
\end{center} 
Si osserva come il primo dei due contenga un'occorrenza del pattern 52413 e il secondo di 25413. Dunque le controimmagini ottenute da questo candidato possono essere semplificate.
\paragraph*{4231:} Anche in questo caso, similmente al caso precedente \`e richiesta la presenza di un'inversione di elementi maggiori di 4 tra 4 e 3 e l'assenza di un'inversione simile tra 4 e 2. Si noti tuttavia come questa eventualit\`a produca necessariamente un pattern 4321, in particolare considerando l'inversione composta da elementi superiori a 4 e gli elementi $3$ e $1$. Dunque il fatto che una permutazione eviti un pattern 4321 comprende anche che essa eviti il pattern trovato e quindi questo candidato non produce controimmagini utili. 
$$Q_{POP}Q^{-1}Av(21) = Av(4321, 25413, 52413)$$
\subsection{Presentazione  dell'algoritmo per la ricerca di controimmagini secondo l'operatore POPstacksort} 
Si ricercano le condizioni che un candidato deve soddisfare per produrre una voluta immagine. Come negli altri casi, si esaminano le inversioni nel candidato, per ogni inversione $(b,a)$ che non viene ordinata da una passata di POPstacksort si verifica una delle sequenti condizioni:
\begin{description}
	\item[a)] un elemento $c>b$ \`e presente tra $b$ e $a$, questo provoca il POP prima che l'algoritmo arrivi ad esaminare $a$;
	\item[b)] un elemento $a^-<a$ (e nessun elemento $c>b$) \`e presente tra $b$ e $a$, quindi $a^-$ viene inserito in coda ed \`e $a$ a provocare il POP;
	\item[c)] tra i due elementi ma non c'\'e nessun elemento minore di $a$ o maggiore $b$ ma c'\'e una coppia ordinata, ovvero esistono ${b^-,a^+}:a<a^+<b^-<b$ e $b\dots{a^+}\dots{b^-}\dots{a}$ \`e una sottosequenza nel candidato.
\end{description}
Quindi per trovare una controimmagine secondo POPstacksort occorre applicare una tra le seguenti decorazioni, per ogni inversione che debba restare tale, al mesh-pattern del candidato:
\begin{center}
\mmpattern{scale=1.5}{2}{1/2,2/1}{}{}$\Rightarrow$
A)\mmpattern{scale=1.5}{2}{1/2,2/1}{}{1/2/2/3/1},
B)\mmpattern{scale=1.5}{2}{1/2,2/1}{1/2}{1/0/2/1/1},
C)\decpatternww{scale=1.5}{2}{1/2,2/1}{}{}{1/0,1/2}{1/1/2/2/{\mpattern{scale=0.4}{2}{1/1,2/2}{}}}{}
\end{center}
Se invece ci si vuole assicurare che un'inversione venga ordinata, la decorazione da apllicare consiste semplicemente nella "negazione" di tutte quelle espresse finora:
\begin{center}
\mmpattern{scale=1.7}{2}{1/2,2/1}{}{}$\Rightarrow$
\decpatternww{scale=1.7}{2}{1/2,2/1}{}{}{1/0,1/2}{}{1/1/2/2/{\mpattern{scale=0.4}{2}{1/1,2/2}{}}}
\end{center}
\paragraph*{Nota:} L'algoritmo, che \`e stato presentato per motivi di completezza, permette di ricavare le controimmagini delle seguenti composizioni: $S_{POP}\circ{S},S_{POP}\circ{Q},S_{POP}\circ{B}$. Questi casi non sono verranno esaminati dato che la loro trattazione richiede troppi dettagli per essere inclusa tra gli argomenti di questa tesi.
\subsection{Composizione di stacksort con POP-queuesort}
Grazie a vari risultati ottenuti applicando i software mostrati nel capitolo 3, \`e stato possibile formulare la seguente congettura:
$$SQ_{POP}^{-1}Av(21) = Av(2431, 4231, 23514, \mmpattern{scale=0.7}{5}{1/3,2/2,3/5,4/1,5/4}{0/4,0/3,0/5,1/3,1/4,1/5}{})$$
Questo equivale a cercare le controimmagini del pattern 231 secondo \textit{Cons}, dato che:
$$SQ_{POP}^{-1}Av(21) = Q_{POP}^{-1}(S^{-1}(Av(21))) = Q_{POP}^{-1}(Av(231))$$
Per dimostrare la congettura occorre dimostrare la doppia inclusione:
\begin{description}
\item[(A)]$Cons^{-1}Av(231) \subseteq Av(2431, 4231, 23514, \mmpattern{scale=0.7}{5}{1/3,2/2,3/5,4/1,5/4}{0/4,0/3,0/5,1/3,1/4,1/5}{})$
\item[(B)]$Av(2431, 4231, 23514, \mmpattern{scale=0.7}{5}{1/3,2/2,3/5,4/1,5/4}{0/4,0/3,0/5,1/3,1/4,1/5}{}) \subseteq Cons^{-1}Av(231)$
\end{description}
\paragraph*{Nota:}Per tutta la dimostrazione i valori che formano i pattern saranno indicati da $a,b,c,d$ ed $e$ (quando presente), $n$ indicher\'a la lunghezza della permutazione $\pi$ e si intende $1\leq{a}<b<c<d<e\leq{n}$.
\paragraph*{Osservazione:} Sia $\pi$ una permutazione in input all'algoritmo \textit{Cons}, tutti gli elementi di $\pi$ che non sono LTR massimi effettuano un bypass. Un LTR massimo pu\`o accodarsi se e solo se \`e il consecutivo dell'elemento in fondo alla coda e provoca un \textit{pop} in tutti gli altri casi.
\begin{center}
\textbf{(A)} $Cons^{-1}Av(231) \subseteq Av(2431, 4231, 23514, \mmpattern{scale=0.7}{5}{1/3,2/2,3/5,4/1,5/4}{0/4,0/3,0/5,1/3,1/4,1/5}{})$
\end{center}
\begin{proof}
Dobbiamo dimostrare che tutte le permutazioni la cui immagine evita il pattern 231 evitano i pattern $2431, 4231, 23514, \mmpattern{scale=0.7}{5}{1/3,2/2,3/5,4/1,5/4}{0/4,0/3,0/5,1/3,1/4,1/5}{}$, ovvero che l'immagine di ogni permutazione che contiene almeno uno di questi pattern contiene un pattern 231.
\paragraph*{2431}
$$\pi = \dots{b}\dots{d}\dots{c}\dots{a}\dots$$
Le non inversioni $(b,d)$ e $(b,c)$ saranno presenti anche nell'immagine e inoltre $c,a$, non essendo LTR massimi, effettuano un bypass. Si osserva che se $b$ effettua un bypass si ha necessariamente un pattern 231 nella sottosequnze $bca$. Se $b$ entra in coda allora $d$ provoca necessariamentee un $pop$ (non pu\`o essere accodato alla stessa coda che contiene $b$, perch\'e la coda \`e consecutiva e $c$, che \`e un valore intermedio tra $b$ e $d$, \`e pi\'u avanti nell'input) e l'immagine contiene la sottoseqenza $bca$.
\paragraph*{4231}
$$\pi = \dots{d}\dots{b}\dots{c}\dots{a}\dots$$
Nessuno degli elementi che formano la sottosequenza $bca$ contenuta in $\pi$ \`e un LTR massimo, quindi bypassano tutti e la sequanza \`e contenuta anche in $Cons(\pi)$.
\paragraph*{23514}$$\pi = \dots{b}\dots{c}\dots{e}\dots{a}\dots{d}\dots$$
Le non inversioni sono presenti anche nell'immagine, in particolare la non inversione $(b,c)$, inoltre $a$ non \`e un LTR massimo quindi effetua un bypass. Si vuole mostrare che la non inversione $(b,c)$ viene sempre inserita nell'output prima di $a$. Se $b,c$ effettuano un bypass si ha sicuramente $bca$ nell'immagine. Se entrambi entrano in coda almeno un pop viene provocato da $e$ prima che si arrivi ad $a$. Anche se $b$ effettua un bypass e $c$ entra in coda quest'ultimo viene sicuramente aggiunto all'output prima di $a$, eventualmente da $e$.   
\subsection*{$\mmpattern{scale=1}{5}{1/3,2/2,3/5,4/1,5/4}{0/4,0/3,0/5,1/3,1/4,1/5}{}$}
Una permutazione che contiene occorrenza di questo pattern ha la forma:
$$\pi = \dots{c}\dots{b}\dots{e}\dots{a}\dots{d}\dots$$dove nessun valore precedente a $b$ \`e maggiore di $c$.\\\\
Si osserva che $b,a$ effettuano un bypass. $c$ \`e un LTR massimo, quindi sia che provochi un pop o meno, verr\'a inserito in coda, dopodich\'e si verifica il bypass di $b$, e $c$ viene sicuramente estratto dalla coda prima di arrivare ad $a$ (eventualmente \`e $e$ a provocare il pop). Si pu\`o escludere che $c$ esca dalla coda prima che $b$ effettui un bypass perch\'e prima di $b$ nessun elemento \`e maggiore di $c$.\\Si forma cos\'i la sottosequenza $bca$.
\end{proof}
\begin{center}
\textbf{(B)} $Av(2431, 4231, 23514, \mmpattern{scale=0.7}{5}{1/3,2/2,3/5,4/1,5/4}{0/4,0/3,0/5,1/3,1/4,1/5}{}) \subseteq Cons^{-1}Av(231)$
\end{center}
\begin{proof}
Dobbiamo dimostrare che tutte le permutazioni che evitano i pattern $2431, 4231, 23514, \mmpattern{scale=0.7}{5}{1/3,2/2,3/5,4/1,5/4}{0/4,0/3,0/5,1/3,1/4,1/5}{}$ sono contenute anche nell'insieme $Cons^{-1}Av(231)$, ovvero si dimostrer\'a che ogni permutazione $\pi$ tale che $231\preceq{Cons}(\pi)$ contiene uno dei pattern elencati.\\
Si osserva che se 3 elementi nell'immagine $Cons(\pi)$ formano un pattern 231, quegli stessi elementi devono formare un pattern 231 o 321 nella controimmagine, dato che \textit{Cons} non produce nuove inversioni. Si analizzano questi due casi separatamente.
\paragraph*{$231\preceq\pi$:}
$$\pi=\dots b\dots c\dots a\dots$$
$$Cons(\pi)=\dots b\dots c\dots a\dots$$
La non inversione $(b,c)$ resta anche nell'imamgine. Si vuole definire l'ordine relativo degli altri elementi di $\pi$ rispetto a $a,b,c$ perch\'e le inversioni $(b,a),(c,a)$ non vengano rimosse da \textit{Cons}.\\\\
Se la forma di $\pi$ \`e tale che $b$ entra in coda, per mentenere la sottosequenza $bca$ \`e necessario che $c$ non sia un LTR massimo. Questa condizione \`e soddisfatta se $\pi$ contiene un'occorrenza del mesh-pattern \mmpattern{scale=0.7}{3}{1/2,2/3,3/1}{}{0/3/2/4/1}, ovvero se contiene uno dei pattern classici 4231 o 2431.\\\\
Se la forma di $\pi$ \`e invece tale per cui $c$ entri in coda, devono essere presenti degli elementi in $\pi$ che provocano un pop prima che $a$ possa bypassare (si noti come queste considerazioni si applicano sia nel caso in cui $b$ sia in coda con $c$ al momento del pop, sia in quelli in cui $b$ \`e gi\`a stato inserito nell'output). Per assicurare il pop \`e sono necessari due nuovi elementi: $e,d$ dove $e$ deve essere posizionato tra $c$ e $a$ mentre $d$ dopo di $e$. L'elemento $e$ assicura il pop se \`e maggiore della testa della coda (se \`e minore tutti gli elementi del pattern effettuano un bypass) ma non pu\`o accodarsi, in quanto $d$ non pu\`o ancora essere in coda e quindi $e$ non pu\`o essere consecutivo alla coda. Queste condizioni sono soddisfatte dal mesh-pattern $\mmpattern{scale=0.7}{4}{1/2,2/3,3/4,4/1}{}{3/3/5/4/1}$, quindi si deve avere che $\pi$ contenga almeno uno tra i pattern classici 23541,23514. Il primo dei due pattern classici viene ignorato, in quanto contiene uno dei pattern gi\`a trovati prima:$2431\preceq23541$.  
\paragraph*{$321\preceq\pi$:}$$\pi=\dots c\dots b\dots a\dots$$$$Cons(\pi)=\dots b\dots c\dots a\dots$$
Ci chiediamo quale deve essere l'ordine relativo degli altri elementi di $\pi$ rispetto a $a,b,c$ per ottenere che l'inversione $(3,2)$ venga ordinata mentre le altre restino invariate.\\\\
Per ordinare $(b,c)$, $c$ deve essere un LTR massimo, in modo che possa entrare nella coda, e non deve verificarsi nessun pop tra $c$ e $b$. Queste condizioni sono rappresentate dalle aree oscurate nel mesh pattern \mmpattern{scale=0.7}{3}{1/3,2/2,3/1}{0/3,1/3}{}. Per forzare un pop tra $c$ e $a$ \`e necessario introdurre due elementi $d,e$ come nel caso precedente: si ottiene cos\'i il mesh-pattern \mmpattern{scale=0.7}{4}{1/3,2/2,3/4,4/1}{0/3,1/3,0/4,1/4}{3/3/5/4/1}, la cui parte decorata permette rappresenta l'unione delle possibili posizioni del valore $d$ nei due successivi mesh-pattern mostrati:
\begin{center}
\mmpattern{scale=1.5}{4}{1/3,2/2,3/4,4/1}{0/3,1/3,0/4,1/4}{3/3/5/4/1} $\Rightarrow$
\mmpattern{scale=1.2}{5}{1/3,2/2,3/5,4/1,5/4}{0/3,1/3,0/4,1/4,0/5,1/5}{},
\mmpattern{scale=1.2}{5}{1/3,2/2,3/5,4/4,5/1}{0/3,1/3,0/4,1/4,0/5,1/5}{} 
\end{center}
Degli ultimi due mesh pattern mostrati, si ignora il secondo in quanto contiene un'occorrenza del pattern 2431, gi\`a trovato in precedenza, nella sottosequenza $beda$.
\end{proof}
\subsection{Composizione di queuesort con POP-queuesort}
Similmente a come fatto per la sezione precedente, si vuole adesso dimostrare il seguente risultato:
$$QQ_{POP}^{-1}Av(21) = Av(4321, 35214, 35241)$$
Questo equivale a cercare le controimmagini del pattern 321 secondo \textit{Cons}, dato che:
$$QQ_{POP}^{-1}Av(21) = Q_{POP}^{-1}(Q^{-1}(Av(21))) = Q_{POP}^{-1}(Av(321))$$
Per dimostrare la congettura occorre dimostrare la doppia inclusione:
\begin{description}
\item[(A)]$Cons^{-1}Av(321) \subseteq Av(4321, 35214, 35241)$
\item[(B)]$Av(4321, 35214, 35241) \subseteq Cons^{-1}Av(321)$
\end{description}
\paragraph*{Nota:} Si adotta la stessa dicitura, per gli elementi che formano i pattern, adottata nella precedente dimsotrazione
\begin{center}
\textbf{(A)} $Cons^{-1}Av(321) \subseteq Av(4321, 35214, 35241)$
\end{center}
\begin{proof}
Dobbiamo dimostrare che tutte le permutazioni la cui immagine evita il pattern 321 evitano i pattern 4321, 35214, 35241.\\
Si procede a dimostrare che l'immagine di ogni permutazione che contiene almeno uno di questi pattern contiene un pattern 321.
\paragraph*{4321:} $$\pi=\dots{d}\dots{c}\dots{b}\dots{a}\dots$$
Nessun memebro della sottosequenza $cba$ \`e un LTR massimo, quindi effettuano tutti un bypass e vengono inseriti nell'output nello stesso ordine, formando un pattern 321.
\paragraph*{35214:}$$\pi=\dots{c}\dots{e}\dots{b}\dots{a}\dots{d}\dots$$
Gli elementi $b,a$ non sono LTR massimi e quindi effettuano un bypass. 
Si vuole dimostrare che l'elemento $c$ viene sempre inserito nell'output prima dell'inversione $(b,a)$. Questo \`e evidente quando $c$ effettua un bypass. 
Se invece $c$ entra in coda, anche se in seguito ad un pop, la testa della coda \`e sicuramente minore di $e$, 
che quindi deve necessariamente causare un $pop$ 
(non pu\`o accodarsi dato che $c$ \`e in coda e $d$ \`e ancora nell'input), 
ponendo $c$ nell'output prima di $b$.
\paragraph*{35241:} In questo caso valgono le stesse osservazioni del caso precedente.
\end{proof}
\begin{center}
\textbf{(A)} $Av(4321, 35214, 35241) \subseteq Cons^{-1}Av(321)$
\end{center}
\begin{proof}
Dobbiamo dimostrare che se una permutazione evita i pattern 4321, 35214, 35241, la sua immagine evita il pattern 321. Si procede a dimostrare che per avere un pattern 321 nell'immagine $Cons(\pi)$ \`e necessario che $\pi$ contenga almeno uno tra i pattern 4321, 35214, 35241.\\\\
Poich\'e \textit{Cons} non produce nuove inversioni gli elementi che $c,b,a$ che formano il pattern 321 nell'immagine devono essere nello stesso ordine relativo anche nella controimmagine.\\\\
Se nessuno degli elementi della sottosequenza $cba$ \`e un LTR massimo allora tutti e 3 gli elementi effettuano un bypass e il pattern 321 \`e preservato. Questo si verifica quando $cba$ sono i 3 elementi minori di un'occorrenza di un pattern 4321.\\\\
Se $c$ invece entra in coda, anche se in seguito ad un pop, deve avvenire un pop prima che $b$ possa effettuare un bypass. Questo si verifica quando un elemento maggiore di $e$ \`e posto tra $c$ e $b$ ed un ulteriore elemento $d$ \`e successivo ad $e$ (come mostrato anche nel caso precedente).
\begin{center}
\mmpattern{scale=1.5}{3}{1/3,2/2,3/1}{}{} $\Rightarrow$
\mmpattern{scale=1.2}{4}{1/3,2/4,3/2,4/1}{}{2/3/5/4/1} 
\end{center}
Una permutazione contiene un'occorrenza del pattern decorato cos\'i ottenuto se contiene uno dei seguenti pattern classici: 35421, 35241, 35214. Di questi il primo viene escluso perch\'e contiene a sua volta un'occorrenza del pattern 4321, gi\`a trovato prima.
\end{proof}
\subsection{Composizione di bubblesort con POP-queuesort}
$$(BQ_{POP})^{-1}(Av(21)) = Q_{POP}^{-1}(B^{-1}(Av(21))) = Q_{POP}^{-1}(Av(231, 321))$$
La classe di pattern ordinabile dalla composizione di bubblesort con Cons \`e data dall'unione dei risultati ottenuti per stacksort e queuesort.\\\\
$$(BQ_{POP})^{-1}(Av(21)) = Av(2431, 4231, 4321, 23514, 35214, 35241, \mmpattern{scale=0.7}{5}{1/3,2/2,3/5,4/1,5/4}{0/4,0/3,0/5,1/3,1/4,1/5}{})$$
Si osserva che l'insieme non \`e minimo, in quanto alcuni pattern contengono istanze di altri: $2413\preceq35214, 4231\preceq35241$. Eliminati i pattern superflui, si ottiene:
$$(BQ_{POP})^{-1}(Av(21)) = Av(2431, 4231, 4321, 23514, \mmpattern{scale=0.7}{5}{1/3,2/2,3/5,4/1,5/4}{0/4,0/3,0/5,1/3,1/4,1/5}{})$$
