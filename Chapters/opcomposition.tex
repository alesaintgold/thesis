\chapter{Composizione di operatori di ordinamento}
\'E molto interessante studiare la combinazione dei vari operatori e le relazioni tra classi di pattern e le loro controimmagini.
\paragraph*{controimmagini} Sia $X$ un operatore di ordinamento, la controimmagine di una certa permutazione $p$, secondo $X$, indicata con $X^{-1}(p)$ rappresenta l'insieme di tutte le permutazioni la cui immagine secondo l'operatore $X$ \'e uguale a $p$.$$X^{-1}(p) = \{\beta : p = X(\beta)\}$$\\
$Av(21)$ \'e l'insieme formato dalle sole permutazioni identit\'a, dato che contiene tutte le permutazioni in cui nessun elemento sia disordinato rispetto ad un altro, ovvero solo le permutazioni crescenti. Si indica con $X^{-1}(Av(21))$ l'insieme di tutte le permutazioni ordinabili da $X$.\\\\
Ad esempio, dato che \'e noto che l'operatore \textit{bubblesort} ordina solo le permutazioni che non contengono pattern 231 e 321\cite{claesson2012sorting}, vale che:$$B^{-1}(Av(21)) = Av(231,321)$$
\paragraph*{Composizione di operatori} Siano due operatori di ordinamento $X$ e $Y$, la loro composizione \'e indicata con $ XY = X \circ Y$, qundi, ad esempio, la composizione di \textit{Stacksort} e \textit{Bubblesort} (in questo ordine) si indica con $SB(\pi) = (S \circ B)(\pi) = S(B(\pi))$.
\paragraph*{Algoritmi per il calcolo di controimmagini} Sono gi\'a stati prodotti alcuni algoritmi per calcolare le controimmagini di pattern secondo gli operaratori \textit{bubblesort}\cite{albert2010inverse}, \textit{stacksort}\cite{claesson2012sorting} e \textit{queuesort}\cite{magnusson2013sorting}\cite{cioni2021characterization}.\\
Questo ci permette, quando si combinano due operatori di ordinamento (di cui di quello applicato per secondo siano noti le classi di pattern da evitare per l'ordinabilit\'a) di cercare per quali pattern l'operatore che viene applicato per primo produce permutazioni che siano ordinabili dal secondo.
\paragraph*{Un caso gi\'a noto: Combinazione di \textit{Stacksort} con \textit{Bubblesort}} Si considera dunque la composione $SB = S\circ{B}$ e ci si chiede quali permutazioni possano essere ordinate da esso.$$(SB)^{-1}Av(21)=B^{-1}S^{-1}(Av(21))=B^{-1}(Av(231))$$
Dunque ricercando per quali permutazioni \textit{bubblesort} evita il pattern 231 si trovano le condizioni per cui una permutazione risulta ordinabile da \textit{SB}.\\
Utilizzando l'algoritmo per le controimmagini di bubblesort\cite{albert2010inverse} si ottiene che:$$(SB)^{-1}(Av(21))=Av(3241, 2341, 4231, 2431)$$
Per ottenere una controimmagine di bubblesort di un pattern classico, si applicano le seguenti regole:
\begin{description}
	\item si cercano tutte le inversioni nell'immagine in esame: poich\'e bubblesort non produce nuove inversioni, tutte le inversioni nell'immagine dovranno gi\'a essere presenti nella controimmagine;
	\item si considera una lista di \textbf{candidati}, composta da ogni pattern minimale che contiene almeno le stesse inversioni;
	\item per ogni candidato, si processa ogni sua coppia disordinata $(b,a)$ nel modo seguente:
	\begin{description}
	\item se $(b,a)$ \'e contenuta anche nell'immagine si ha che $b$ non \'e un LTR massimo o che lo \'e ma c'\'e un altro LTR massimo tra $b$ e $a$, quindi si decora la coppia come segue, producendo un nuovo pattern:
	\begin{center}
		\mmpattern{scale=1.4}{2}{1/2,2/1}{}{}$\Rightarrow$\mmpattern{scale=1.4}{2}{1/2,2/1}{}{0/2/2/3/1}
	\end{center}
	\item se $(b,a)$ \'e un'inversione del candidato che non \'e nell'immagine allora la coppia viene ordinata da una passata di bubblesort, si ha che $b$ \'e l'ultimo LTR massimo prima di $a$, la coppia viene decorata nel modo seguente:
	\begin{center}
		\mmpattern{scale=1.4}{2}{1/2,2/1}{}{}$\Rightarrow$\mmpattern{scale=1.4}{2}{1/2,2/1}{0/2,1/2}{}\\
	\end{center}
	\end{description}
	\item applicare ad un candidato le condizioni per tutte le sue inversioni (considerando che se una zona \'e oscurata da una condizione, non pu\'o essere decorata da un altra) fornisce un mesh-pattern decorato che descrive il pattern contenuto nella controimmaigne.
\end{description}
\subsection*{Esempio pratico di calcolo di una controimmagine}
A titolo d'esempio verr\'a mostrato come si pu\'o arrivare al risultato (gi\'a noto\cite{albert2010inverse}) riguardo alla combinazione $(S\circ B)$:
$$(SB)^{-1}(Av(21)) = Av(2341, 2431, 3241, 4231)$$
Prima di tutto si ricerca di quale classe di pattern nonch\'e tramite quale operatore occorre calcolare le controimmagini:
$$(SB)^{-1}(Av(21)) = B^{-1}(S^{-1}(Av(21)))=B^{-1}(Av(231))$$
Dato che, come noto\cite{limbrief} $S^{-1}(Av(21))=Av(231)$.\\
I pattern candidati ad essere controimmagini di 231 sono i pattern che contengono le coppie $(2,1),(3,1)$, ovvero i pattern 231,321.
\paragraph*{231:} si osserva che una condizione per l'inversione $(2,1)$ \'e gi\'a realizzata dall'elemento $3$, applicando le regole descritte prima all'inversione $(3,1)$ si ha il mesh-pattern \mmpattern{scale=0.7}{3}{1/2,2/3,3/1}{}{0/3/3/4/1}, ovvero i pattern 2341, 2431, 4231.
\paragraph*{321:} in questo caso si ha l'inversione $(3,2)$ che deve essere invertita, quindi si inseriscono le aree oscurate, in seguito si osserva che, come prima, $3$ realizza la condzione per l'inversione $(2,1)$ e dunque basta aggiungere le decorazioni per l'inversione $(3,1)$. Di seguito si mostrano i passaggi:
\begin{center}
\mmpattern{scale=1.5}{3}{1/3,2/2,3/1}{}{}$\Rightarrow$
\mmpattern{scale=1.5}{3}{1/3,2/2,3/1}{0/3,1/3}{}$\Rightarrow$
\mmpattern{scale=1.5}{3}{1/3,2/2,3/1}{0/3,1/3}{2/3/3/4/1}\\
\end{center}
Si ottiene cos\'i il pattern 3241 che, unito agli altri fornisce esattamente il risultato atteso.\\\\
Si osservi che il software \texttt{permutasort}, presentato nel capitolo 3, pu\'o aiutare ad arrivare alle stesse conclusioni. Infatti lanciando il programma sull'operatore SB e con $n=4$ si ottengono i seguenti risultati:\\\\\texttt{\$ python permutasort.py 4 SB }\\\dots\\\texttt{The following 4 4-permutations are not sortable with the operator SB:}\\\texttt{(2, 3, 4, 1)}\\\texttt{(2, 4, 3, 1)}\\\texttt{(3, 2, 4, 1)}\\\texttt{(4, 2, 3, 1)}\\\dots\\\\
Questo tuttavia \'e utile come indicazione, ma non \'e sufficiente per essere sicuri del risultato: infatti non si pu\'o escludere la presenza di pattern classici pi\'u lunghi n\'e la presenza di pattern barrati.\\
Per quanto quindi questo strumento si sia rivelato molto utile per avere un'idea di quali pattern ricercare, soprattutto durante le combinazioni pi\'u complesse che presenteremo in seguito, non pu\'o sostituire del tutto un approccio pi\'u formale e teorico, come l'applicazione di uno degli algoritmi introdotti prima.
\section*{Combinazione di queuesort con bubblesort}
$$(QB)^{-1}(Av(21))=B^{-1}Q^{-1}(Av(21))=B^{-1}(Av(321))$$
Si osserva che le inversioni in $321$ sono $(3,2),(3,1),(2,1)$ e l'unico pattern minimo che le contiene tutte \'e appunto $321$, che quindi \'e l'unico candidato.\\
In questo caso a tutte le inversioni deve essere applicata la prima condizione, a parte a $(2,1)$ perch\'e la condzione \'e gi\'a soddisfatta da 3. Il pattern decorsato che si ottiene \'e:
\begin{center}
\mmpattern{scale=1.5}{3}{1/3,2/2,3/1}{}{0/3/3/4/{\dots\dots1},0/3/2/4/1}\\
\end{center}
Ovvero  l'unione di \mmpattern{scale=0.7}{3}{1/3,2/2,3/1}{}{0/3/2/4/1} e \mmpattern{scale=0.7}{3}{1/3,2/2,3/1}{}{0/3/3/4/1}. Tuttavia questi due pattern possono essere semplificati nel primo: si nota infatti che \mmpattern{scale=0.7}{3}{1/3,2/2,3/1}{}{0/3/2/4/1}, che si ottiene dalle condizioni per $(3,2)$ soddisfa anche le condizioni per $(3,1)$ e inoltre vale che \mmpattern{scale=0.7}{3}{1/3,2/2,3/1}{}{0/3/2/4/1}$\preceq$\mmpattern{scale=0.7}{3}{1/3,2/2,3/1}{}{0/3/2/4/1,2/3/3/4/1}.\\
Quindi le preimmagini del pattern 321 tramite bubblesort sono date da:
\begin{center}
\mmpattern{scale=1.4}{3}{1/3,2/2,3/1}{}{0/3/2/4/1}.
\end{center}
Questo mesh pattern corrisponde ai pattern classici $4321,3421$.$$QB^{-1}(Av(21))=Av(4321,3421)$$
\section*{Composizioni di un operatore di ordinamento con Stacksort}
L'algoritmo per le controimmagini di \textit{stacksort} \'e abbastanza simile a quello per \textit{bubblesort}.\\
Anche in questo caso si osservano tutte le inversioni contenute nell'immagine in esame e se ne trae una lista di pattern minimali candidati ad essere controimmagini e li si esaminano ad uno ad uno.\\
Per ogni coppia disordinata $(b,a)$ presente nel candidato: 
\begin{description}
\item se $(b,a)$ \'e presente anche nell'immagine allora deve essere presente un elemento $c>b$ tra $b$ e $a$ che fa uscire $b$ dalla pila prima che $a$ vi entri
\begin{center}
\mmpattern{scale=1}{2}{1/2,2/1}{}{} $\Longrightarrow$\mmpattern{scale=1}{2}{1/2,2/1}{}{1/2/2/3/1} 
\end{center}
\item se invece $(b,a)$ non \'e presente allora si pu\'o escludere la presenza di tale elemento
\begin{center}
\mmpattern{scale=1}{2}{1/2,2/1}{}{} $\Longrightarrow$\mmpattern{scale=1}{2}{1/2,2/1}{1/2}{} 
\end{center}
\end{description}
\subsection*{Composizione  di {queuesort} con {stacksort}}
$$(QS)^{-1}(Av(21))=S^{-1}Q^{-1}(Av(21))=S^{-1}(Av(321))$$
Si ricercano dunque le controimmagini di $321$ secondo \textit{stacksort}.\\$321$ \'e l'unico candidato, quindi sappiamo che tutte le controimmagini devono contenere il pattern $321$.\\
L'elemento $3$ deve entrare nella pila ed uscirne prima del $2$, per fare ci\'o deve esserci un elemento $3^+$ maggiore di $3$ tra $3$ e $2$. Similmente \'e necessario un elemento $2^+>2$ tra $2$ e $1$ per assicurare che $2$ esca dalla pila prima che $1$ vi entri.\\
\begin{center}
\mmpattern{scale=1.5}{3}{1/3,2/2,3/1}{}{1/3/2/4/1,2/2/3/4/1}
\end{center}
Dunque le controimmagini che cerchiamo devono essere nella forma $33^+22^+1$.\\
Se $2^+<3$ allora la controimmagine assume la forma del pattern $45231$.\\
Altrimenti se $2^+>3$, si possono ottenere due diverse controimmagini:
\begin{description}
	\item[$3^+>2^+$] genera il pattern $35241$
	\item[$2^+>3^+$] genera il pattern $34251$
\end{description}
$$(QS)^{-1}(Av(21)) = Av(34251, 35241, 45231)$$
\subsection*{Composizione di {bubblesort} con {stacksort}}
$$(BS)^{-1}(Av(21))=S^{-1}B^{-1}(Av(21))=S^{-1}(Av(231,321))$$
Gi\'a dall'analisi della combinazione precedente \'e risultato che $S^{-1}(Av(321))=Av(34251, 35241, 45231)$ qundi \'e necessario calcolare solo $S^{-1}(Av(231))$. Quest'ultimo risultato \'e stato ampiamente studiato in letteratura, in quanto analogo al caso di una variante di \textit{stacksort} che utilizza 2 pile. Il risultato che si ottiene \'e dunque che $S^{-1}(Av(231))=Av(2341, 3\overline{5}241)$\cite{claesson2012sorting}.\\\\
Seguendo l'algoritmo applicato finora si osserva che i pattern candidati per le controimmagini sono $231,321$.\\\\
Si uniscono dunque i due risultati:
$$S^{-1}(Av(231))=Av(2341,3\overline{5}241), S^{-1}(Av(321))=Av(34251, 35241, 45231)$$
Si osserva che $2341\preceq 34251$, quindi $34251$ non \'e minimo.\\
Inoltre i pattern $35241, 3\overline{5}241$, possono essere rappresentati dal pattern minimo $3241$ che rende non minimo anche $34251$.\\Il risultato che si ottiene \'e:$$(BS)^{-1}(Av(21))=Av(2341,3241,45231)$$
\section*{Combinazioni di un operatore di ordinamento con \textit{queuesort}}
Nonostante anche per \textit{queuesort} sia stato trovato un algoritmo per le controimmagini\cite{magnusson2013sorting}, approcciare il problema "manualmente" analizzando i possibili comportamenti di \textit{queuesort} rispetto alle diverse possibili permutazioni risulta essere pi\'u semplice e pi\'u comprensibile.\\\\
Se nel valutare il comportamento di \textit{queuesort} rispetto ad una controimmagine se ci si vuole assicurare che un elemento entri nella coda allora lo si vuole "forzare" ad essere un LTR maxima, dunque si applicher\'a il seguente mesh-pattern all'elemento in considerazione:
\begin{center}
\mmpattern{scale=1.5}{1}{1/1}{0/1}{}
\end{center}
Al contrario, se si ha a che fare con un elemento che risulta essere un right-to-lrft-maxima ma lo si vuole forzare a svuotare la coda, si avr\'a il seguente risultato:
\begin{center}
\mmpattern{scale=1.5}{1}{1/1}{}{0/1/1/2/1}
\end{center}
Per forzare un elemento a effettuare un bypass \'e necessario che esso sia minore del primo elemento della coda, per questa condizione \'e pi\'u difficile formalizzare una startegia, ma si vedranno in seguito esempi di come pu\'o essere fatto. 
\subsection*{Composizione di {stacksort} con {queuesort}}
$$(SQ)^{-1}(Av(21))=Q^{-1}S^{-1}(Av(21))=Q^{-1}(Av(231))$$Ancora una volta si considerano come candidati i pattern minimali che contengano almeno le inversioni $(2,1),(3,1)$ ovvero $231,321$.\\\\
Se gli elementi del pattern $231$ nell'immagine formano un pattern $321$ nella controimmagine allora il valore $3$ deve entrare nella coda. Sia che $2$ effettui un bypass sia che provochi l'estrazione di alcuni elementi della coda comunque esso viene aggiunto all'output prima che $3$ venga estratto. Si rende necessario un elemento $3^+>3$ posto tra $2$ e $1$ per estarre il $3$ dalla coda e inserirlo nell'input prima dell'$1$, tuttavia $3^+$ non deve essere maggiore dell'elemento in fondo alla coda, altrimenti viene accodato. Si teorizza l'esistenza di un elemento $4>3^+$ posto tra $3$ e $2$, che rappresenta il massimo della coda quando si raggiunge $3^+$.\\
Il pattern che si ottiene deve essere nella forma $3423^+1$, ovvero $35241$\\\\
se invece gli elementi del pattern $231$ nell'immagine formano un pattern $231$ anche nella controimmagine che lo contiene a sua volta \'e necessario che $2$ effettui un bypass o che venga estratto dalla coda prima che $3$ vi entri. Il che implica la presenza di un valore $2^+>2$ prima di $2$ o tra $2$ e $3$.\\Similmente $3$ deve essere aggiunto all'output prima di $1$, qundi deve bypassare o essere estratto prima che $1$ bypassi. Il che implica l'esistenza di un elemento $4$ in posizione precdente a $1$; in particolare i pattern minimali che rispettano queste condizoni sono quelli in cui $4$ compare prima di $2$ e si ha che $2^+=4$.\\
\begin{center}
\mmpattern{scale=1.5}{3}{1/2,2/3,3/1}{}{0/3/2/4/1}
\end{center}
Gli unici pattern minimali che soddisfano queste condizioni sono $2431,4231$.
\\\\Si nota inoltre che, il pattern calcolato prima $35241\in Q^{-1}(Av(321))$ risulta adesso superfluo in quanto $4231\preceq35241$
$$(SQ)^{-1}(Av(21))=Av(2431,4231)$$
\subsection*{Composizione di {bubblesort} con {queuesort}}
$$(BQ)^{-1}(Av(21))=Q^{-1}B^{-1}(Av(21))=Q^{-1}(Av(231,321))$$
Ancora una volta, si osserva che $Q^{-1}(Av(231))$ \'e gi\'a stato calcolato nel caso precedente, e resta dunque da unire i risultati gi\'a ottenuti con $Q^{-1}(Av(321))$.\\
Gli elementi del pattern $321$ nell'immagine devono essere a sua volta nello stesso odrine nell'immagine.\\
In particolare gli elementi $3,2$ devono effettuare un bypass o essere inseriti nella coda ed estratti prima dell'elemento successivo.\\\\
\begin{center}
\mmpattern{scale=1.5}{3}{1/3,2/2,3/1}{}{0/3/2/4/1}
\end{center}
Il pattern $4321$ sicuramente soddisfa queste condizioni: se $4$ bypassa allora tutti gli elementi del pattern, essendo minori bypassano a loro volta, se viene inserito in coda ed estratto allora bypassano comunque, se viene inserito in coda e non estratto allora tutti gli elementi successivi evitano la coda e vengono aggiunti all'output, eventualmente dopo aver provocato l'estrazione di alcuni elementi dalla coda.\\
Se $3$ viene aggiunto in coda ed estratto allora si ha una situazione $3421$, dove $2$ bypassa.\\
$$(SQ)^{-1}(Av(21))=Av(2431,3421,4231,4321)$$
\section*{Contenitori POP}
Lo studio di combinazioni di operatori che utilizzano contenitori POP si rivela pi\'u difficile, in quando si tratta di casi non approfonditi in letteratura.\\
Sono tuttavia noti i pattern che rendono una permutazione non ordinabile, come presentato nel capitolo 2, ovvero:
$$S_{POP}^{-1}(Av(21)) = Av(231,312)$$$$Q_{POP}^{-1}(Av(21))=Av(321,2413)$$
Grazie a questi dati, \'e possibile individuare un caso di facile risoluzione: ovvero quello in cui un algoritmo che usa contenitori POP \'e concatenato ad un operatore regolare.
$$(X_{POP}\circ{Y})(\pi)= X_{POP}(Y(\pi)))\Rightarrow(X_{POP}\circ{Y})^{-1}(\pi) = Y^{-1}X_{POP}^{-1}(\pi)=Y^{-1}(Av(m))$$
Sapendo l'insieme di pattern $m$ che rendono la permutazione non ordinabile dall'operatore POP baster\'a cercare le loro controimmagini secondo l'operatore regolare con i metodi usati finora, come verr\'a mostrato nelle sezioni seguenti.
\section*{POP Queuesort}L'algoritmo POP queuesort richiede una diversa analisi da POP stacksort.\\
Esistono diverse versioni di \textit{POP queuesort} ed in particolare ne esistono due ottimali\cite{cioni2021sorting}: \textit{Min} e \textit{Cons}.
\begin{description}
	\item[\textit{Min}] in questa versione l'operazione di POP viene eseguita solo se il primo elemento della coda \'e il successivo dell'ultimo elemento aggiunto all'output; se l'elemento in input \'e maggiore dell'ultimo elemento della coda (o se la coda \'e vuota) allora viene accodato mentre negli altri casi, se l'elemento dell'input \'e minore della testa della coda allora bypassa altrimenti si esegue un POP.
	\item[\textit{Cons}] questa versione si basa sull'idea di avere sempre elementi consecutivi nella coda; \'e la versione che verr\'a adottata in questa tesi; da qui in avanti ogni riferimento a \textit{POP-queuesort} sar\'a riferito a \textit{Cons}
\end{description}
Nell'algoritmo \textit{Cons} si definisce una coda POP come un insieme $Q$, inizialmente vuoto, di interi su cui \'e possibile fare le seguenti operazioni:
\begin{description}
	\item[Enqueue($\pi_i$, $Q$):] inserisce l'$i$-esimo elemento della permutazione $\pi$ in $Q$, accodandolo alla coda;
	\item[Pop($Q$):] estrae gli elementi contenuti in $Q$ e li aggiunge all'output nello stesso ordine a cui sono stati aggiunti a $Q$;
	\item[Bypass($\pi_i$):] posiziona l'$i$-esimo elemento di $\pi$ nell'output.
\end{description}
Inoltre si utilizzano le seguenti funzioni:
\begin{description}
	\item[Back($Q$):] restituisce il valore dell'ultimo elemento aggiunto a $Q$
	\item[Front($Q$):] restituisce il valore dell'elemento in $Q$ che vi \'e stato inserito per primo
\end{description}
\begin{algorithm}[H]
   \caption{Cons - POP Queuesort}
\begin{algorithmic}[1]
\State $Q\leftarrow\emptyset$ 
\For{$i=1$ {\bfseries to} $n$}
	\If{$Q=\emptyset$ \textbf{or} $\pi_i=Back(Q)+1$}
		\State Enqueue($\pi_i$)
	\Else
		\If{ $Front(Q)>\pi_i$}
			\State $Bypass(\pi_i)$
		\Else
			\State $Pop(Q)$
			\State $Enqueue(\pi_i)$
		\EndIf
	\EndIf
\EndFor
\If{$Q\neq\emptyset$}
\State $Pop(Q)$
\EndIf
\end{algorithmic}
\end{algorithm}
La dimostrazione che \textit{Cons} (cos\'i come \textit{Min}) sia un algoritmo ottimale nella classe degli algoritmi \textit{POP-queuesort} si ha dal fatto che esso ordina tutte e sole le sequenze dell'insieme Av(321, 2413), che sono esattamente tutte e sole le permutazioni ordinabili con una POP queue\cite{cioni2021sorting}.
\subsection*{Composizione di {POP-stacksort} con {stacksort}}
$$(S_{POP}\circ{S})^{-1}(Av(21))=S^{-1}(Av(231,312))$$
Si \'e gi\'a calcolato che $S^{-1}(Av(231))=Av(2341, 3\overline{5}241)$, quindi manca da calcolare le controimmagini di $312$.\\
I pattern candidati che possono generare $312$ sono quelli che contengono le coppie $(3,1),(3,2)$, ovvero $321, 312$.
Perch\'e gli elementi di un pattern $321$ nella controimmagine generino un pattern $312$ tramite \textit{stacksort} \'e necessario che $3$ entri in pila e ne esca prima che $2$ vi entri, deve quindi essere presente un valore $4$ posizionato tra $3$ e $2$ che provochi l'uscita di $3$, ovvero il pattern $3421$.\\
Se gli elementi del pattern $312$ nell'immagine formano lo stesso pattern anche nella controimmagine allora anche in questo caso bisogna assicurarsi che $3$ venga estratto dalla pila prima di $1$, si ha quindi il pattern $3412$.\\Si uniscono tutti i risultati:
$$(S_{POP}S)^{-1}(Av(21))=Av(2341, 3412, 3421, 3\overline{5}241)$$
\subsection*{Composizione di {POP-stacksort} con {bubblesort}}
$$(S_{POP}\circ{B})^{-1}(Av(21))=B^{-1}(Av(231,312))$$
Anche in questo caso la controimmagine di $231$ \'e stata calcolata: $B^{-1}(Av(231)) = (S\circ{B})^{-1}(Av(21)) = Av(2341, 2431, 3241, 4231)$.\\
Si ricerca dunque le preimmaigini di 312 secondo bubblesort. \\
Gli elementi del pattern $312$ nell'immagine possono solo formare un pattern $312$ o $321$ nella controimmagine.\\
Si pu\'o osservare come gli stessi elementi di un pattern $321$ non possono divenire un $312$. La presenza dell'elemento maggiore del pattern all'inizio di essa evita l'ordinamento degli elementi successivi.\\
Perch\'e un pattern $312$ rimanga invariato \'e sufficiente la presenza di un elemento maggiore di tutto il pattern in posizione tale da evitare l'ordinamento di $3$ con qualsiasi elemento, applicando le regole descritte prima per \textit{bubblesort}: 
\begin{center}
\mmpattern{scale=1.5}{3}{1/3,2/1,3/2}{}{0/3/2/4/1}$=4312,3412$
\end{center}
$$(S_{POP}{B})^{-1}(Av(21))=Av(2341, 2431, 3241, 3412, 4231, 4312)$$
\subsection*{Composizione di {POP-stacksort} con {queuesort}}
$$(S_{POP}\circ{Q})^{-1}(Av(21))=Q^{-1}(Av(231,312))$$$$Q^{-1}(Av(231)=Av(4231,2431)$$
Possibili controimmagini di $312: 312, 321$.\\
Se la controimmagine contiene $312$ bisogna far si che $3$ effettui un bypass o che esca prima di $1$, in ogni caso queste condizioni sono soddisfatte dal seguente mesh-pattern, $4312$:
\begin{center}
\mmpattern{scale=1.5}{3}{1/3,2/1,3/2}{}{1/2/2/3/1}
\end{center}
$$(S_{POP}{Q})^{-1}(Av(21))=Av(2431,4231,4312)$$
\subsection*{Composizione di {POP-queuesort} con {stacksort}}
$$(Q_{POP}{S})^{-1}(Av(21)) = S^{-1}(Q_{POP}^{-1}(Av(21))) = S^{-1}(Av(321, 2413))$$
Come in molti altri casi la controimmagine di 321 \'e stata calcolata:$$S^{-1}(Av(321)) = (QS)^{-1}(Av(21)) = Av(34251, 35241, 45231)$$
Si ricerca dunque le controimmagini di 2413 secondo stacksort, applicando lo stesso algoritmo dei casi precedenti.\\
Le inversioni di 2413 sono $(2,1),(4,1),(4,3)$, che definiscono l'insieme di candidati 2413, 2431, 4213, 4231.
\paragraph*{2431} I primi due elementi del pattern sono ordinati fra loro, dunque per lasciare il pattern invariato \'e sufficiente evitare che le coppie 41 e 43 vengano ordinate. Come gi\'a visto questo si ottiene posizionando in mezzo ai due elementi della coppia un elemnto maggiore di entrambi. Si ottiene cos\'i  il pattern 24513.
\paragraph*{2431} Le inversioni di questa controimmagine sono quelle gi\'a individute nell'immagine pi\'u $(3,1)$. Quindi applicando le condizioni per le controimmagini si ottiene il seguente meta pattern:
\begin{center}\mmpattern{scale=1}{4}{1/2,2/4,3/3,4/1}{}{}$\Rightarrow$\mmpattern{scale=1}{4}{1/2,2/4,3/3,4/1}{3/3,3/4}{2/4/3/5/1}\end{center}
La controimmagine di 2413 che contiene 2431 \'e quindi 24531.
\paragraph*{4213}\begin{center}\mmpattern{scale=1}{4}{1/4,2/2,3/1,4/3}{}{}$\Rightarrow$\mmpattern{scale=1}{4}{1/4,2/2,3/1,4/3}{1/4}{2/4/3/5/1}\\La controimmagine di 2413 che contiene 4213 \'e 42513.\end{center}
\paragraph*{4231}\begin{center}\mmpattern{scale=1}{4}{1/4,2/2,3/3,4/1}{}{}$\Rightarrow$\mmpattern{scale=1}{4}{1/4,2/2,3/3,4/1}{1/4,3/3,3/4}{2/4/3/5/1}\\La controimmagine di 4213 \'e 42531.\end{center}
Unendo tutti i risultati si ottiene:$$(Q_{POP}\circ{S})^{-1}(Av(21)) = Av(24513, 24531, 34251, 35241, 42513, 42531, 45231)$$
\subsection*{Composizione di {POP-queuesort} con {bubblesort}}
$$(Q_{POP}{B})^{-1}(Av(21)) = B^{-1}Q_{POP}^{-1}(Av(21))=B^{-1}(Av(321,2413))$$
La classe di pattern $B^{-1}(Av(321))$ \'e gi\'a stata calcolata:
$$B^{-1}(Av(321)) = QB^{-1}(Av(21)) = Av(3421, 4321)$$
Si ricerca adesso la classe di pattern corrispondente a $B^{-1}(Av(2413))$. La lista dei candidati \'e la seguente: 2413, 2431, 4213, 4231, 4321.
\paragraph*{Osservazione:} i pattern 2431, 4231, 4321 contengono l'inversione $(3,1)$ che non presente nel pattern dell'immagine. Questa specifica inversione \'e in posizione tale che nessuno dei due elementi sia un LTR massimo, e dunque l'inversione non possa essere ordinata. In questi 3 pattern infatti, l'elemento 4 si trova all'interno dell'area oscurata che si dovrebbe aggiungere durante l'applicazione dell'algortimo per le preimmagini di bubblesort all'inversione $(3,1)$. Ne consegue che i pattern  2431, 4231, 4321, pur essendo nella lista di candidati non possono generare il pattern 2413.\\
Si ricercano ora le preimmagini date dagli altri candidati:
\paragraph*{2413:}Si osserva che 4 soddisfa gi\'a la condizione per l'inversione $(2,1)$, dunque si prende l'intersezione tra le condizioni per le inversioni $(4,3),(4,1)$ ottenendo il seguente pattern decorato:
\begin{center}
\mmpattern{scale=1.2}{4}{1/2,2/4,3/1,4/3}{}{0/4/3/5/1}
\end{center}
Da questo si ottengono i pattern classici 52413, 25413, 24513.
\paragraph*{4213:} L'inversione $(4,2)$ deve essere ordinata, in quanto non presente in 2413.
\begin{center}
\mmpattern{scale=1.2}{4}{1/4,2/2,3/1,4/3}{}{} $\Rightarrow$
\mmpattern{scale=1.2}{4}{1/4,2/2,3/1,4/3}{0/4,1/4}{}
\end{center}
Si osserva poi che 4 soddsfa gi\'a le condizioni per l'inversione $(2,1)$, mentre applicando le condizioni per le altre inversioni, tenendo conto dell'area oscurata si ottiene:
\begin{center}
\mmpattern{scale=1.2}{4}{1/4,2/2,3/1,4/3}{0/4,1/4}{2/4/4/5/1}
\end{center}
Corrispondente ai pattern classici 42513, 42153.\\
Si uniscono i risultati ottenuti.
$$(Q_{POP}{B})^{-1}(Av(21)) = Av(3421, 4321, 24513, 25413, 42153, 42513, 52413)$$
\subsection*{Composizione di {POP-queuesort} con {queuesort}}
$$(Q_{POP}{Q})^{-1}(Av(21)) = Q^{-1}(321, 2413)$$
La controimmagine $Q^{-1}(Av(321))$ \'e gi\'a stata trovato nel caso $B\circ{Q}$, e vale che $Q^{-1}(Av(321)) = Av(4321)$.\\
Se si applica le regole per la ricerca di preimamgini di queuesort a 2413 si ottiene:
\begin{center}\mmpattern{scale=1.5}{4}{1/2,2/4,3/1,4/3}{}{0/4/2/5/1}\end{center}
Quindi si ottiene che $(Q_{POP}\circ{Q})^{-1}(Av(21)) = Av(4321, 25413, 52413)$.
\section*{Presentazione  dell'algoritmo per la ricerca di controimmagini secondo l'operatore POPstacksort}
Si ricercano le condizioni che un candidato deve soddisfare per produrre una voluta immagine. Come negli altri casi, si esaminano le inversioni nel candidato, per ogni coppia disordinata $(b,a)$ che non viene ordinata da una passata di POPstacksort si verifica una delle sequenti condizioni:
\begin{description}
	\item[a)] un elemento $c>b$ \'e presente tra $b$ e $a$, questo provoca il POP prima che l'algoritmo arrivi ad esaminare $a$;
	\item[b)] un elemento $a^-<a$ (e nessun elemento $c>b$) \'e presente tra $b$ e $a$, quindi $a^-$ viene inserito in coda ed \'e $a$ a provocare il POP;
	\item[c)] tra i due elementi ma non c'\'e nessun elemento minore di $a$ o maggiore $b$ ma c'\'e una coppia ordinata, ovvero esistono ${b^-,a^+}:a<a^+<b^-<b$ e $b\dots{a^+}\dots{b^-}\dots{a}$ \'e una sottosequenza nel candidato.
\end{description}
Quindi per trovare una controimmagine secondo POPstacksort occorre applicare le seguenti decorazioni, per ogni coppia disordinata che debba restare tale, al meta pattern del candidato:
\begin{center}
\mmpattern{scale=1.5}{2}{1/2,2/1}{}{}$\Rightarrow$
A)\mmpattern{scale=1.5}{2}{1/2,2/1}{}{1/2/2/3/1},
B)\mmpattern{scale=1.5}{2}{1/2,2/1}{1/2}{1/0/2/1/1},
C)\decpatternww{scale=1.5}{2}{1/2,2/1}{}{}{1/0,1/2}{1/1/2/2/{\mpattern{scale=0.4}{2}{1/1,2/2}{}}}{}
\end{center}
Se invece ci si vuole assicurare che una coppia disordinata venga ordinata, la decorazione da apllicare consiste semplicemente nella "negazione" di tutte quelle espresse finora:
\begin{center}
\mmpattern{scale=1.7}{2}{1/2,2/1}{}{}$\Rightarrow$
\decpatternww{scale=1.7}{2}{1/2,2/1}{}{}{1/0,1/2}{}{1/1/2/2/{\mpattern{scale=0.4}{2}{1/1,2/2}{}}}
\end{center}
\subsection*{Composizione di {stacksort} con {POP-stacksort}}
$$(S{S_{POP}})^{-1}(Av(21)) = S_{POP}^{-1}(S^{-1}(Av(21))) = S_{POP}^{-1}(Av(231))$$
\\Si applicano le condizioni per le controimmagini ai due candidati: 231 e 321.\\\\
Nel caso di 231, le inversioni sono $(2,1),(3,1)$. Si osserva che 3 gi\'a soddisdfa la condzione A per la coppia $(2,1)$, quindi resta da applicare le condizioni alla coppia $(3,1)$:
\begin{center}
\mmpattern{scale=1.5}{3}{1/2,2/3,3/1}{}{2/3/3/4/1},
\mmpattern{scale=1.5}{3}{1/2,2/3,3/1}{2/3}{2/0/3/1/1},
\decpatternww{scale=1.5}{3}{1/2,2/3,3/1}{}{}{2/0,2/3}{2/1/3/3/{\mpattern{scale=0.4}{2}{1/1,2/2}{}}}{}
\end{center}
Dalla condizione A si ottiene il pattern 2341, dalla B il 3412 dalla condizione C si ottengono, considerando la posizione relativa degli elemeneti nella coppia aggiuntiva rispetto all'elemento 2, i pattern 25341, 35241, 45231.\\\\
Se gli elementi del pattern 231 nell'immagine invece formano un pattern 321 nella controimmagine, allora si applica la condizione per ordinare una coppia alla coppia $(3,2)$:
\begin{center}
\decpatternww{scale=2}{3}{1/3,2/2,3/1}{}{}{1/0,1/1,1/3}{}{1/2/2/3/{\mpattern{scale=0.5}{2}{1/1,2/2}{}}}
\end{center}
Si applicano adesso le condizioni per la coppia $(3,1)$ che ci si aspetta resti disordinata:
\begin{center}
\decpatternww{scale=2}{3}{1/3,2/2,3/1}{}{}{1/0,1/1,1/3}{2/3/3/4/1}{1/2/2/3/{\mpattern{scale=0.5}{2}{1/1,2/2}{}}}
\decpatternww{scale=2}{3}{1/3,2/2,3/1}{}{}{1/0,1/1,1/3,2/3}{2/0/3/1}{1/2/2/3/{\mpattern{scale=0.5}{2}{1/1,2/2}{}}}
\decpatternww{scale=2}{3}{1/3,2/2,3/1}{}{}{1/0,1/1,1/3,2/3,2/0}{2/1/3/3/{\mpattern{scale=0.5}{2}{1/1,2/2}{}}}{1/2/2/3/{\mpattern{scale=0.5}{2}{1/1,2/2}{}}}
\end{center}
Si osservi come applicando le condzioni A e B per la coppia $(3,1)$ queste siano gi\'a soddisfatte per la coppia $(2,1)$. Applicando la condizione C la coppia pu\'o essere costruita utilizzando l'elemento 2 come membro minore all'interno della coppia, e questo soddisfa anche la condizione A per la coppia $(2,1)$. Si ottengono dunque i pattern 3241, 4312, 4231.\\\\
Si osserva che, tra i risultati ottenuti prima, alcuni risultano non necessari: $2341\preceq25341, 3241\preceq35241, 4231\preceq45231$. Si semplificano e si ottiene il seguente risultato:
$$(S{S_{POP}})^{-1}(Av(21)) = Av(2341, 3241, 3412, 4231, 4312)$$
\subsection*{Composizione di {queuesort} con {POP-stacksort}}
$$Q{S_{POP}}^{-1}(Av(21)) = S_{POP}^{-1}(Q^{-1}(Av(21))) = S_{POP}^{-1}(Av(321))$$
\\Si applicano anche stavolta le condizioni per la ricerca di controimmagini al pattern 321. Essendo 321 una sequenza decrementale, non ci sono altri candidati; le sue inversioni sono $(3,2),(3,1),(2,1)$ ed \'e necessario applicare tutte le combinazioni possibili di condizioni A,B,C a queste coppie.\\\\
Una volta applicate le diverse condizioni alla coppia $(3,2)$ \'e evidente che queste soddisfano anche le condizioni per la coppia $(3,1)$:
\begin{description}
	\item applicando la condizione A e ponendo quindi un elemento maggiore di 3 tra 3 e 2, allora lo stesso elemento rappresenta un valore maggiore di 3 posizionato tra 3 e 1 (condzione A per la coppia $(3,1)$)
	\item se si applica la condizione B, ponendo un valore $2^-<2$ tra 2 e 3, allora o si ha che $2^-<1$, che soddisfa la condizione B per la coppia $(3,1)$, o che $1<2^-<2$, e allora la coppia $(2^-,2)$ \'e una coppia ordinata tra 3 e 1 (condizione C per ls coppia $(3,1)$)
	\item nel caso, infine, in cui si applichi la condizione C e si ponga quindi una coppia ordinata tra 3 e 2 allora la stessa coppia soddisfa la condizione C per la coppia $(3,1)$.
\end{description}
In sintesi, i seguenti pattern soddisfano le condzioni da imporre alle coppie $(3,2),(3,1)$:
\begin{center}
\mmpattern{scale=1.7}{3}{1/3,2/2,3/1}{}{1/3/2/4/1}
\mmpattern{scale=1.7}{3}{1/3,2/2,3/1}{1/3}{1/0/2/2/1}
\decpatternww{scale=1.7}{3}{1/3,2/2,3/1}{}{}{2/0,2/2,2/3}{2/1/3/2/{\mpattern{scale=0.5}{2}{1/1,2/2}{}}}{}
\end{center}
Una volta applicate le condizioni anche alla coppia $(2,1)$ si ottengono i seguenti 9 meta-pattern, da cui si ottengono 22 pattern classici:
\begin{center}
\mmpattern{scale=1.7}{3}{1/3,2/2,3/1}{}{1/3/2/4/1,2/2/3/4/1}
\mmpattern{scale=1.7}{3}{1/3,2/2,3/1}{2/3,2/2}{1/3/2/4/1,2/0/3/1/1}
\decpatternww{scale=1.7}{3}{1/3,2/2,3/1}{1/3/2/4/1}{}{2/0,2/2,2/3}{2/1/3/2/{\mpattern{scale=0.5}{2}{1/1,2/2}{}}}{}
\mmpattern{scale=1.7}{3}{1/3,2/2,3/1}{1/3}{1/0/2/2/1,2/2/3/4/1}
\mmpattern{scale=1.7}{3}{1/3,2/2,3/1}{2/2,2/3}{1/0/2/2/1,2/0/3/1/1}
\decpatternww{scale=1.7}{3}{1/3,2/2,3/1}{}{}{1/3,2/3,2/2,2/0}{1/0/2/2/1,2/1/3/2/{\mpattern{scale=0.5}{2}{1/1,2/2}{}}}{}
\decpatternww{scale=1.7}{3}{1/3,2/2,3/1}{}{}{1/0,1/1,1/3}{2/2/3/4/1,1/2/2/3/{\mpattern{scale=0.5}{2}{1/1,2/2}{}}}{}
\decpatternww{scale=1.7}{3}{1/3,2/2,3/1}{}{}{1/0,1/1,1/3,2/2,2/3}{2/0/3/1/1,1/2/2/3/{\mpattern{scale=0.5}{2}{1/1,2/2}{}}}{}
\decpatternww{scale=1.7}{3}{1/3,2/2,3/1}{}{}{1/0,1/1,1/3,2/0,2/2,2/3}{1/2/2/3/{\mpattern{scale=0.5}{2}{1/1,2/2}{}},2/1/3/2/{\mpattern{scale=0.5}{2}{1/1,2/2}{}}}{}
\end{center}
Scrivendo come pattern classici quelli che si ottengono da questi meta-patterns ed eliminando quelli ridondanti, si ottiene il seguente risultato:
\begin{eqnarray*}
&Q{S_{POP}}(Av(21)) = Av(34251, 35241, 41352, 42351, 45231, \\
& 45312, 51342, 51423, 52341, 52413, 53412)
\end{eqnarray*}
\subsection*{Composizione di {bubblesort} con {POP-stacksort}}
$$(B\circ{S}_{POP})^{-1}(Av(21)) = S_{POP}^{-1}(B^{-1}(Av(21))) = S_{POP}^{-1}(Av(231,321))$$
Il l'insieme minimo di pattern per l'ordinabilit\'a di permutazione con la composzione bubblesort, POPstacksort \'e l'unione degli insiemi dei due risultati precedenti, stacksort e queuesort.\\
Si nota, tuttavia che ogni pattern contenuto nell'insieme ottenuto dalla combinazione con queuesort contiene \textbf{almeno} un pattern tra quelli ottenuti da stacksort. Nello specifico, a titolo esemplificativo:$$3241\preceq34251, 35241, 41352, 42351$$$$3412\preceq53412$$$$4231\preceq45231, 51342, 52341, 52413$$$$4312\preceq45312, 51423$$ 
Dunque l'insieme minimo di pattern che caratterizza le permutazioni ordinabili da $B\circ{S}_{POP}$ \'e lo stesso di $S\circ{S}_{POP}$.
$$(B{S_{POP}})^{-1}(Av(21)) = Av(2341, 3241, 3412, 4231, 4312)$$
\paragraph*{Nota:} Non si ha a disposizione un algoritmo per la ricerca di controimmagini di pattern tramite \textit{Cons}. Per questo motivo le classi di pattern ordinabili dalle combinazioni mostrate di seguito, ovvero quelle in cui si compone POP-queuesort ad un altro algoritmo di ordinamento, verrano ricercate con un metodo diverso da quello adottato finora: si presenteranno delle congetture su quali classi di pattern che si ritengono possano rispondere alle nostre condizioni, e si verificher\'a che l'insieme di permutazioni ordinabili e l'insieme di permutazioni contenute nella classe di pattern coincidano.
\subsection*{Composizione di stacksort con POP-queuesort}
Grazie a vari risultati ottenuti applicando i software mostrati nel capitolo 3, \'e stato possibile formulare la seguente congettura:
$$SQ_{POP}^{-1}Av(21) = Av(2431, 4231, 23514, \mmpattern{scale=0.7}{5}{1/3,2/2,3/5,4/1,5/4}{0/4,0/3,0/5,1/3,1/4,1/5}{})$$
Questo equivale a cercare le controimmagini del pattern 231 secondo \textit{Cons}, dato che:
$$SQ_{POP}^{-1}Av(21) = Q_{POP}^{-1}(S^{-1}(Av(21))) = Q_{POP}^{-1}(Av(231))$$
Per dimostrare la congettura occorre dimostrare la doppia inclusione:
\begin{description}
\item[(A)]$Cons^{-1}Av(231) \subseteq Av(2431, 4231, 23514, \mmpattern{scale=0.7}{5}{1/3,2/2,3/5,4/1,5/4}{0/4,0/3,0/5,1/3,1/4,1/5}{})$
\item[(B)]$Av(2431, 4231, 23514, \mmpattern{scale=0.7}{5}{1/3,2/2,3/5,4/1,5/4}{0/4,0/3,0/5,1/3,1/4,1/5}{}) \subseteq Cons^{-1}Av(231)$
\end{description}
\paragraph*{Nota:}Per tutta la dimostrazione i valori che formano i pattern saranno indicati da $a,b,c,d$ ed $e$ (quando presente), $n$ indicher\'a la lunghezza della permutazione $\pi$ e si intende $1\leq{a}<b<c<d<e\leq{n}$.
\paragraph*{Osservazione:} Sia $\pi$ una permutazione in input all'algoritmo \textit{Cons}, tutti gli elementi di $\pi$ che non sono LTR massimi effettuano un bypass. Un LTR massimo pu\'o accodarsi se e solo se \'e il consecutivo dell'elemento in fondo alla coda e provoca un \textit{pop} in tutti gli altri casi.
\begin{center}
\textbf{(A)} $Cons^{-1}Av(231) \subseteq Av(2431, 4231, 23514, \mmpattern{scale=0.7}{5}{1/3,2/2,3/5,4/1,5/4}{0/4,0/3,0/5,1/3,1/4,1/5}{})$
\end{center}
\begin{proof}
Dobbiamo dimostrare che tutte le permutazioni la cui immagine evita il pattern 231 evitano i pattern $2431, 4231, 23514, \mmpattern{scale=0.7}{5}{1/3,2/2,3/5,4/1,5/4}{0/4,0/3,0/5,1/3,1/4,1/5}{}$, ovvero che l'immagine di ogni permutazione che contiene almeno uno di questi pattern contiene un pattern 231.
\paragraph*{2431}
$$\pi = \dots{b}\dots{d}\dots{c}\dots{a}\dots$$
Le non inversioni $(b,d)$ e $(b,c)$ saranno presenti anche nell'immagine e inoltre $c,a$, non essendo LTR massimi, effettuano un bypass. Si osserva che se $b$ effettua un bypass si ha necessariamente un pattern 231 nella sottosequnze $bca$. Se $b$ entra in coda allora $d$ provoca necessariamentee un $pop$ (non pu\'o essere accodato alla stessa coda che contiene $b$, perch\'e la coda \'e consecutiva e $c$, che \'e un valore intermedio tra $b$ e $d$, \'e pi\'u avanti nell'input) e l'immagine contiene la sottoseqenza $bca$.
\paragraph*{4231}
$$\pi = \dots{d}\dots{b}\dots{c}\dots{a}\dots$$
Nessuno degli elementi che formano la sottosequenza $bca$ contenuta in $\pi$ \'e un LTR massimo, quindi bypassano tutti e la sequanza \'e contenuta anche in $Cons(\pi)$.
\paragraph*{23514}$$\pi = \dots{b}\dots{c}\dots{e}\dots{a}\dots{d}\dots$$
Le non inversioni sono presenti anche nell'immagine, in particolare la non inversione $(b,c)$, inoltre $a$ non \'e un LTR massimo quindi effetua un bypass. Si vuole mostrare che la non inversione $(b,c)$ viene sempre inserita nell'output prima di $a$. Se $b,c$ effettuano un bypass si ha sicuramente $bca$ nell'immagine. Se entrambi entrano in coda almeno un pop viene provocato da $e$ prima che si arrivi ad $a$. Anche se $b$ effettua un bypass e $c$ entra in coda quest'ultimo viene sicuramente aggiunto all'output prima di $a$, eventualmente da $e$.   
\subsection*{$\mmpattern{scale=1}{5}{1/3,2/2,3/5,4/1,5/4}{0/4,0/3,0/5,1/3,1/4,1/5}{}$}
Una permutazione che contiene occorrenza di questo pattern ha la forma:
$$\pi = \dots{c}\dots{b}\dots{e}\dots{a}\dots{d}\dots$$dove nessun valore precedente a $b$ \'e maggiore di $c$.\\\\
Si osserva che $b,a$ effettuano un bypass. $c$ \'e un LTR massimo, quindi sia che provochi un pop o meno, verr\'a inserito in coda, dopodich\'e si verifica il bypass di $b$, e $c$ viene sicuramente estratto dalla coda prima di arrivare ad $a$ (eventualmente \'e $e$ a provocare il pop). Si pu\'o escludere che $c$ esca dalla coda prima che $b$ effettui un bypass perch\'e prima di $b$ nessun elemento \'e maggiore di $c$.\\Si forma cos\'i la sottosequenza $bca$.
\end{proof}
\begin{center}
\textbf{(B)} $Av(2431, 4231, 23514, \mmpattern{scale=0.7}{5}{1/3,2/2,3/5,4/1,5/4}{0/4,0/3,0/5,1/3,1/4,1/5}{}) \subseteq Cons^{-1}Av(231)$
\end{center}
\begin{proof}
Dobbiamo dimostrare che tutte le permutazioni che evitano i pattern $2431, 4231, 23514, \mmpattern{scale=0.7}{5}{1/3,2/2,3/5,4/1,5/4}{0/4,0/3,0/5,1/3,1/4,1/5}{}$ sono contenute anche nell'insieme $Cons^{-1}Av(231)$, ovvero si dimostrer\'a che ogni permutazione $\pi$ tale che $231\preceq{Cons}(\pi)$ contiene uno dei pattern elencati.\\
Si osserva che se 3 elementi nell'immagine $Cons(\pi)$ formano un pattern 231, quegli stessi elementi devono formare un pattern 231 o 321 nella preimmagine, dato che \textit{Cons} non produce nuove inversioni. Si analizzano questi due casi separatamente.
\paragraph*{$231\preceq\pi$:}
$$\pi=\dots b\dots c\dots a\dots$$
$$Cons(\pi)=\dots b\dots c\dots a\dots$$
La non inversione $(b,c)$ resta anche nell'imamgine. Si vuole definire l'ordine relativo degli altri elementi di $\pi$ rispetto a $a,b,c$ perch\'e le inversioni $(b,a),(c,a)$ non vengano rimosse da \textit{Cons}.\\\\
Se la forma di $\pi$ \'e tale che $b$ entra in coda, per mentenere la sottosequenza $bca$ \'e necessario che $c$ non sia un LTR massimo. Questa condizione \'e soddisfatta se $\pi$ contiene un'occorrenza del mesh-pattern \mmpattern{scale=0.7}{3}{1/2,2/3,3/1}{}{0/3/2/4/1}, ovvero se contiene uno dei pattern classici 4231 o 2431.\\\\
Se la forma di $\pi$ \'e invece tale per cui $c$ entri in coda, devono essere presenti degli elementi in $\pi$ che provocano un pop prima che $a$ possa bypassare (si noti come queste considerazioni si applicano sia nel caso in cui $b$ sia in coda con $c$ al momento del pop, sia in quelli in cui $b$ \'e gi\'a stato inserito nell'output). Per assicurare il pop \'e sono necessari due nuovi elementi: $e,d$ dove $e$ deve essere posizionato tra $c$ e $a$ mentre $d$ dopo di $e$. L'elemento $e$ assicura il pop se \'e maggiore della testa della coda (se \'e minore tutti gli elementi del pattern effettuano un bypass) ma non pu\'o accodarsi, in quanto $d$ non pu\'o ancora essere in coda e quindi $e$ non pu\'o essere consecutivo alla coda. Queste condizioni sono soddisfatte dal mesh-pattern $\mmpattern{scale=0.7}{4}{1/2,2/3,3/4,4/1}{}{3/3/5/4/1}$, quindi si deve avere che $\pi$ contenga almeno uno tra i pattern classici 23541,23514. Il primo dei due pattern classici viene ignorato, in quanto contiene uno dei pattern gi\'a trovati prima:$2431\preceq23541$.  
\paragraph*{$321\preceq\pi$:}$$\pi=\dots c\dots b\dots a\dots$$$$Cons(\pi)=\dots b\dots c\dots a\dots$$
Ci chiediamo quale deve essere l'ordine relativo degli altri elementi di $\pi$ rispetto a $a,b,c$ per ottenere che l'inversione $(3,2)$ venga ordinata mentre le altre restino invariate.\\\\
Per ordinare $(b,c)$, $c$ deve essere un LTR massimo, in modo che possa entrare nella coda, e non deve verificarsi nessun pop tra $c$ e $b$. Queste condizioni sono rappresentate dalle aree oscurate nel mesh pattern \mmpattern{scale=0.7}{3}{1/3,2/2,3/1}{0/3,1/3}{}. Per forzare un pop tra $c$ e $a$ \'e necessario introdurre due elementi $d,e$ come nel caso precedente: si ottiene cos\'i il mesh-pattern \mmpattern{scale=0.7}{4}{1/3,2/2,3/4,4/1}{0/3,1/3,0/4,1/4}{3/3/5/4/1}, la cui parte decorata permette rappresenta l'unione delle possibili posizioni del valore $d$ nei due successivi mesh-pattern mostrati:
\begin{center}
\mmpattern{scale=1.5}{4}{1/3,2/2,3/4,4/1}{0/3,1/3,0/4,1/4}{3/3/5/4/1} $\Rightarrow$
\mmpattern{scale=1.2}{5}{1/3,2/2,3/5,4/1,5/4}{0/3,1/3,0/4,1/4,0/5,1/5}{},
\mmpattern{scale=1.2}{5}{1/3,2/2,3/5,4/4,5/1}{0/3,1/3,0/4,1/4,0/5,1/5}{} 
\end{center}
Degli ultimi due mesh pattern mostrati, si ignora il secondo in quanto contiene un'occorrenza del pattern 2431, gi\'a trovato in precedenza, nella sottosequenza $beda$.
\end{proof}
\subsection*{Composizione di queuesort con POP-queuesort}
Similmente a come fatto per la sezione precedente, si vuole adesso dimostrare il seguente risultato:
$$QQ_{POP}^{-1}Av(21) = Av(4321, 35214, 35241)$$
Questo equivale a cercare le controimmagini del pattern 321 secondo \textit{Cons}, dato che:
$$QQ_{POP}^{-1}Av(21) = Q_{POP}^{-1}(Q^{-1}(Av(21))) = Q_{POP}^{-1}(Av(321))$$
Per dimostrare la congettura occorre dimostrare la doppia inclusione:
\begin{description}
\item[(A)]$Cons^{-1}Av(321) \subseteq Av(4321, 35214, 35241)$
\item[(B)]$Av(4321, 35214, 35241) \subseteq Cons^{-1}Av(321)$
\end{description}
\paragraph*{Nota:} Si adotta la stessa dicitura, per gli elementi che formano i pattern, adottata nella precedente dimsotrazione
\begin{center}
\textbf{(A)} $Cons^{-1}Av(321) \subseteq Av(4321, 35214, 35241)$
\end{center}
\begin{proof}
Dobbiamo dimostrare che tutte le permutazioni la cui immagine evita il pattern 321 evitano i pattern 4321, 35214, 35241.\\
Si procede a dimostrare che l'immagine di ogni permutazione che contiene almeno uno di questi pattern contiene un pattern 321.
\paragraph*{4321:} $$\pi=\dots{d}\dots{c}\dots{b}\dots{a}\dots$$
Nessun memebro della sottosequenza $cba$ \'e un LTR massimo, quindi effettuano tutti un bypass e vengono inseriti nell'output nello stesso ordine, formando un pattern 321.
\paragraph*{35214:}$$\pi=\dots{c}\dots{e}\dots{b}\dots{a}\dots{d}\dots$$
Gli elementi $b,a$ non sono LTR massimi e quindi effettuano un bypass. 
Si vuole dimostrare che l'elemento $c$ viene sempre inserito nell'output prima dell'inversione $(b,a)$. Questo \'e evidente quando $c$ effettua un bypass. 
Se invece $c$ entra in coda, anche se in seguito ad un pop, la testa della coda \'e sicuramente minore di $e$, 
che quindi deve necessariamente causare un $pop$ 
(non pu\'o accodarsi dato che $c$ \'e in coda e $d$ \'e ancora nell'input), 
ponendo $c$ nell'output prima di $b$.
\paragraph*{35241:} In questo caso valgono le stesse osservazioni del caso precedente.
\end{proof}
\begin{center}
\textbf{(A)} $Av(4321, 35214, 35241) \subseteq Cons^{-1}Av(321)$
\end{center}
\begin{proof}
Dobbiamo dimostrare che se una permutazione evita i pattern 4321, 35214, 35241, la sua immagine evita il pattern 321. Si procede a dimostrare che per avere un pattern 321 nell'immagine $Cons(\pi)$ \'e necessario che $\pi$ contenga almeno uno tra i pattern 4321, 35214, 35241.\\\\
Poich\'e \textit{Cons} non produce nuove inversioni gli elementi che $c,b,a$ che formano il pattern 321 nell'immagine devono essere nello stesso ordine relativo anche nella controimmagine.\\\\
Se nessuno degli elementi della sottosequenza $cba$ \'e un LTR massimo allora tutti e 3 gli elementi effettuano un bypass e il pattern 321 \'e preservato. Questo si verifica quando $cba$ sono i 3 elementi minori di un'occorenza di un pattern 4321.\\\\
Se $c$ invece entra in coda, anche se in seguito ad un pop, deve avvenire un pop prima che $b$ possa effettuare un bypass. Questo si verifica quando un elemento maggiore di $e$ \'e posto tra $c$ e $b$ ed un ulteriore elemento $d$ \'e successivo ad $e$ (come mostrato anche nel caso precedente).
\begin{center}
\mmpattern{scale=1.5}{3}{1/3,2/2,3/1}{}{} $\Rightarrow$
\mmpattern{scale=1.2}{4}{1/3,2/4,3/2,4/1}{}{2/3/5/4/1} 
\end{center}
Una permutazione contiene un'occorrenza del pattern decorato cos\'i ottenuto se contiene uno dei seguenti pattern classici: 35421, 35241, 35214. Di questi il primo viene escluso perch\'e contiene a sua volta un'occorrenza del pattern 4321, gi\'a trovato prima.
\end{proof}
\subsection*{Composizione di bubblesort con POP-queuesort}
$$(BQ_{POP})^{-1}(Av(21)) = Q_{POP}^{-1}(B^{-1}(Av(21))) = Q_{POP}^{-1}(Av(231, 321))$$
La classe di pattern ordinabile dalla composizione di bubblesort con Cons \'e data dall'unione dei risultati ottenuti per stacksort e queuesort.\\\\
$$(BQ_{POP})^{-1}(Av(21)) = Av(2431, 4231, 4321, 23514, 35214, 35241, \mmpattern{scale=0.7}{5}{1/3,2/2,3/5,4/1,5/4}{0/4,0/3,0/5,1/3,1/4,1/5}{})$$
Si osserva che l'insieme non \'e minimo, in quanto alcuni pattern contengono istanze di altri: $2413\preceq35214, 4231\preceq35241$. Eliminati i pattern superflui, si ottiene:
$$(BQ_{POP})^{-1}(Av(21)) = Av(2431, 4231, 4321, 23514, \mmpattern{scale=0.7}{5}{1/3,2/2,3/5,4/1,5/4}{0/4,0/3,0/5,1/3,1/4,1/5}{})$$
