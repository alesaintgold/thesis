\chapter{Algoritmi di odrinamento e classi di pattern}
In questo capitolo verranno introdotti i principali algoritmi di ordinamento che utilizzano sorting devices, in particolare stack-sort,
queue-sort e bubble sort, e altri concetti necessari per l'analisi della loro composizione.\\\\
Durante l'esecuzione questi algoritmi possono 
salvare gli elementi in un contenitore (la diversa struttura dati
adottata definise i diversi algoritmi) dalla quale poi vengono prelevati
per essere aggiunti all'output.\\\\
Una sola iterazione non garantisce l'ordinamento della permutazione,
dunque gli algoritmi devono essere iterati pi\'u volte, ogni volta sul risultato della iterazione precedente. In ogni caso alla fine delle i-esima
iterazione i maggiori $i$ elementi avranno raggiunto la loro posizione finale, 
dunque sono necessari al massimo $n-1$ iterazioni per ordinare la permutazione.\\\\
Essendo interessati al comportamento di una sola iterazione di questi
algoritmi si esaminer\'a un operatore, definito appositamente per ogni algoritmo, che descrive la singola iterazione. \\
Ad esempio, prendendo l'algoritmo bubble-sort si far\'a riferimento all'operatore $B(\pi)$, dove $\pi$ \'e una permutazione di
interi, tale che n iterazioni del bubble-sort possano essere rappresentate
da $B^n(\pi) = B(\dots B(\pi)\dots)$.
\section*{Bubble sort}
L'algoritmo di ordinamento bubble-sort prevede di scorrere gli elementi da 
ordinare dal primo al penultimo, ed ogni volta confrontare ogni elemento 
con il suo successivo per scambiarli se non sono ordinati.\\
Il risultato di una singola iterazione di bubble-sort su una permutazione $\pi=\pi_1\pi_2\dots\pi_n$ \'e calcolato dall'operatore $B(\pi)$. Per una permutazione $\pi$ con valore massimo $n$ vale che $\pi = \pi_Ln\pi_R$, allora $B(\pi) = B(\pi_L)\pi_rn$.
\begin{algorithm}[H]
   \caption{$B(\pi)$}
\begin{algorithmic}[1]
   \For{$i=1$ {\bfseries to} $n-1$}
   \If{$\pi_i > \pi_{i+1}$} 
   \State Swap $\pi_i$ and $\pi_{i+1}$
   \EndIf
   \EndFor
\end{algorithmic}
\end{algorithm}
\section*{Stack sort}
L'operatore $S(\pi)$ rappresenta il risultato ottenuto applicando un'iterazione di stack sort su una permutazione $\pi$.\\
Il primo passo consiste nell'inserire $\pi_1$ nella pila. Poi lo si confronta con l'elemento $\pi_2$. Se $\pi_1>\pi_2$ allora il secondo viene messo nella pila sopra $\pi_1$, altrimenti $\pi_1$ viene estratto dalla pila e inserito nell'output e $\pi_2$ viene inserito nella pila.\\
Gli stessi passi vengono eseguiti per tutti gli altri elementi presenti nell'input, se viene trovato un elemento nell'input maggiore dell'elemento in cima alla pila, la pila viene svuotata finch\'e questa condizione non diviene falsa, poi l'elemento viene spinto nella pila.\\
Finiti gli elementi nell'input, se necessario, si svuota completamente la pila nell'output.\cite{limbrief}\\
Sia $\pi = \pi_Ln\pi_R$, con $n$ valore massimo in $\pi$, vale che $S(\pi)=S(\pi_L)S(\pi_R)n$
\begin{algorithm}[H]
   \caption{operatore S - stack sort, singola iterazione}
\begin{algorithmic}[1]
\State initialize an empty stack
   \For{$i=1$ {\bfseries to} $n$}
   \While{ stack is unempty \textbf{and} $\pi_i>$top of the stack}
   \State pop from the stack to the output
   \EndWhile
   \State push($\pi_i$)
   \EndFor
   \State emtpy the stack in the output
\end{algorithmic}
\end{algorithm}
\section*{Queue sort}
Per ogni elemento $\pi_i$ della permutazione $\pi$ in input se la coda \'e vuota o il suo ultimo elemento \'e minore di $\pi_i$, si accoda $\pi_i$, altrimenti si tolgono elementi dalla coda ponendoli nell'output fino a che l'elemento davanti  alla coda non \'e maggiore di $\pi_i$, poi si aggiunge $\pi_i$ all'output. Si svuota la coda nell'output\cite{magnusson2013sorting}. 
\begin{algorithm}[H]
   \caption{operatore Q - queue sort, singola iterazione}
\begin{algorithmic}[1]
\State initialize an empty queue
\For{$i=1$ {\bfseries to} $n$}
\If{empty queue \textbf{or} last in queue $<\pi_i$}
\State enqueue($\pi_i$)
\Else
\While{first in queue $<\pi_i$}
\State dequeue($\pi_i$)
\EndWhile
\State add $\pi_i$ to the output
\EndIf
\EndFor
\State emtpy the queue in the output
\end{algorithmic}
\end{algorithm}
\paragraph*{Bypass} L'operazione che pone un elemento nell'output senza passare dal contenitore si dice \textbf{bypass}. Questa viene svolta normalmente nel queuesort, ma talvolta pu\'o essere introdotta in altri algoritmi.
\paragraph*{Osservazione}\textit{Bubble sort} \'e un caso particolare sia di \textit{queue sort} che di \textit{stack sort}.\\
Se infatti si fissa a 1 la dimensione della pila o della coda dei rispettivi operatori il comportamento che questi assumono \'e quello di una cella che, scorrendo l'input, contiene sempre il massimo valore trovato, mentre gli altri vengono messi nell'output.
\paragraph*{left-to-right maxima} In una sequenza i \textit{LTR maxima} sono gli elementi che risultano essere maggiori di ogni altro elemento che li precede. Ad esempio nella sequenza 142387596 i \textit{LTR maxima} sono 1,4,8,9.\\\\
La nozione di \textit{LTR maxima} risulta molto importante per lo studio di alcuni degli algoritmi di ordinamento presentati. In particolare durante l'esecuzione di \textit{queuesort} i LTR maxima sono tutti e soli gli elementi che entrano nella coda, mentre durante l'esecuzione di \textit{bubblesort} sono gli unici che vengono scambiati; in entrambi gli algoritmi l'ordine relativo degli altri elementi non viene alterato.
\paragraph*{Contenitori POP}Un caso di studio interessante \'e quello in cui i contenitori di stack sort e queue sort vengano sostituiti dalla loro versione POP, ovvero un contenitore con politiche di estrazione e inserimento analoghe ma quando viene eseguita un estrazione il contenitore viene svuotato completamente. Si definiscono con questa variante, gli algoritmi \textbf{pop-stacksort} e \textbf{pop-queuesort} 
\section*{Classi di pattern di permtuazioni}
\cite{bouvel2022preimages}
Siano $\alpha,\beta$ due sequenze di interi, si indica con $\alpha \subseteq \beta$ che $\alpha$ \'e una sottosequenza di $\beta$, anche se non necessariamente una sottosequenza consecutiva.\\\\
Si dice che una permutazione $\delta$ \'e un pattern contenuto in una permutazione $\tau$ se esiste una sottosequenza di $\tau$ di ordine isomorfico rispetto a $\delta$, e si indica con $\delta\preceq\tau$. Ad esempio $24153$ contiene il pattern $312$ perch\'e $413\subset{24153}$.
\paragraph*{Pattern Standard}
Una \textbf{standardizzazione}\cite{claesson2012sorting} di una sequenza di numeri \'e un altra sequenza della stessa lunghezza in cui l'elemento minore della sequenza originale \'e stato sostituito da 1, il secondo minore con un 2, ecc.\\
Ad esempio, la standardizzazione di 5371 \'e 3241.
\paragraph*{Classi di pattern} La relazione di sottopermutazione \'e una relazione di ordine parziale che viene studiata con dei sottoinsiemi chiamati \textbf{pattern di classi}. Ogni classe di pattern $D$ pu\'o essere caratterizzata dall'insieme minimo $M$ che evita:$$ D = Av(M) = \{\beta:\mu\not\preceq\beta\forall\mu\in M\}$$
\paragraph*{Pattern barrati} I pattern classici esprimono in quale relazione di ordine devono essere gli elementi di una sequenza. In alcuni casi pu\'o rivelarsi necessario dover esprimere informazioni ulteriori, come l'assenza di un elemento in una data posizone. Questa informazione pu\'o essere fornita grazie all'utilizzo dei pattern barrati.\\
Un pattern barrato \'e un pattern in cui ogni numero pu\'o essere barrato.
\\La barra che viene posta sopra ad un elemento di un pattern indica che se una sequenza contiene un elemento in quella posizone allora essa non contiene il pattern.\\Ad esempio $1\overline{4}23\not\preceq 162534$, dato che $1423 \preceq 2534$, mentre $1\overline{4}23\preceq146235$ dato che $123\preceq146,235$
\section*{Algoritmi e pattern-avoidance}
\paragraph*{pattern 231}Una permutazione $\pi$ contiene un pattern $231$ se $231\preceq\pi$, ovvero se $\exists a<b<c: \pi =\dots{b}\dots{c}\dots{a}\dots$.\\\\
\'E noto \cite{limbrief} che una permutazione pu\'o essere ordinata da una sola passata di stack sort se e solo se non contiene pattern 231.\\\\
Allo stesso modo, sono note simili condizioni perch\'e una permutazione possa essere ordinata da una sola passata degli altri operatori descritti precedentemente.\\
\begin{center}
\begin{tabular}{ |c|c| } 
\hline
\textbf{Operatore} & \textbf{Permutazioni ordinabili con una sola passata} \\ 
\hline
Stack sort & Av(231)\\ 
Queue sort & Av(321)\\ 
Bubble sort & Av(231,321)\\ 
Pop-stack sort & Av(231,312)\\ 
Pop-queue sort& Av(321,2413)\\ 
\hline
\end{tabular}
\end{center}
\section*{Mesh-pattern, pattern barrati e pattern decorati}
In questa sezione verranno brevemente introdotti i mesh-pattern\cite{branden2011mesh} attraverso degli esempi; le macro per la realizzazione delle griglie sono state prodotte dalla \textit{Reykjavik University}\cite{patternmacros}.
\begin{center}
\mpattern{scale=1.4}{3}{1/1,2/3,3/2}{0/2,1/2,2/2}
\end{center}
Il pattern mostrato nella griglia soprastante appare in una permutazione se si pu\'o individuare un pattern $132$ posizionato in modo tale che le zone oscurate non sono occupate da altri elementi della permutazione.\\\\
La sequenza $1243$ contiene un'istanza del pattern, dato che gli elementi $1,4,3$ formano un pattern $132$ e l'elemento $2$ si trova in una zona non oscurata.\\
La sequenza $3142$ invece non contiene lo stesso pattern, dato che nonostante $1,4,2$ formano un pattern $132$, l'elemento 3 si trova nella zona oscurata $(0,2)$ (le zone oscurate si individuano indicando la posizione del loro angolo in basso a sinistra, numerandoli da $0$).\\\\
Nel seguente pattern l'area bianca indicata con $1$ indica che per avere un'istanza di questo pattern in una permutazione, all'interno dell'area deve essere presente almeno un altro elemento della permutazione: 
\begin{center}
\mmpattern{scale=1.4}{3}{1/1,2/3,3/2}{3/0,3/1,3/2}{2/3/3/4/1}
\end{center}
Questo pattern \'e contenuto, ad esempio, nelle sequenze $1342$, $12453$ o $13524$.