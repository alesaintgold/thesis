\chapter{Programmi realizzati}
\section*{PermutaSort}
Il primo programma che ho realizzato viene lanciato da linea di comando specificando come argomenti un intero $n$ e un operatore di ordinamento $X$. Il suo scopo \'e quello di enumerare tutte le $n$-permutazioni ordinabili e non ordinabili con una sola passata di $X$, stampa anche i possibili risultati di $X$ su $n$-permutazioni.\\\\
La classe \texttt{SelectorPermutations} \'e il core del programma, viene inizializzata passando un intero e un operatore al costruttore. Grazie alla funzione \texttt{itertools.permutations} vengono generate le $n$-permutazioni e , scorrendole tutte, vi si applica l'operatore $X$. Il risultato viene aggiunto alla lista \texttt{outcomes} e viene valutato: se \'e ordinato la permutazione da cui si \'e ottenuto viene aggiunta alla lista \texttt{sortable}, altrimenti a \texttt{unsortable}.\\La classe fornisce dei \textit{getters} per le liste e non ha altri metodi computativi, oltre al costruttore.\\
Dopo essere stata instanziata la classe viene usata come riferimento per i dati che ha gi\'a calcolato nel costruttore, e non produce ulteriori risultati.\\Di seguito viene mostrato il listato della classe:
\\\lstinputlisting{./python/permutasort.py}
\pagebreak
\section*{Pattfinder}
Questo programma ha lo scopo di selezionare tra tutte le permutazioni di una data dimensione quali contengono un dato pattern classico quali no.\\\\
Il programma viene lanciato da linea di comando prendendo come argomenti un intero e una sequenza di interi consecutivi eventualemente non ordinati.\\
L'intero rappresenta la lunghezza delle permutazioni da scorrere e la sequenza rappresenta il pattern da ricercare.\\
Gli argomenti vengono passati al costruttore della classe \texttt{PatternAvoid}, l'intero $n$ viene utilizzato per generare le $n$-permutazioni grazie alla funzione \texttt{permutations}. Per ogni permutazione viene controllato se contiene o no il pattern e viene inserita nella rispettiva lista: \texttt{containing} o \texttt{notcontaining}.\\\\
La classe presenta i getter per le liste prodotte e due ulteriori metodi: 
\begin{description}
\item[\texttt{patternize}] serve a \textit{standardizzare} una sequenza, si ottiene la versione ordinata della sequenza, e si sostituisce ogni valore con la posizione che occupa una volta oridnato;  
\item[\texttt{contains}] verifica che una sequenza contenga un pattern o meno, grazie alla funzione \texttt{itertools.combinations} si ottengono tutte le sottosequenze (anche non consecutive), le si standardizzano e le si confrontano con il pattern da ricercare.
\end{description}
\lstinputlisting{./python/pattfinder.py}
%\section*{Script per la verifica dei pattern}