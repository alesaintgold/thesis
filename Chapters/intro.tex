\chapter{Introduzione}
Gli algoritmi di ordinamento sono uno degli argomenti pi\`u classici dell'informatica.\\
In questa tesi si esamina una classe di questi algoritmi, ovvero quelli che utilizzano dei contenitori: strutture dati in cui vengono salvati alcuni elementi della sequenza da ordinare. Il pi\`u noto tra questi algoritmi \`e bubblesort.\\\\
Nel caso di bubblesort, il primo elemento della sequenza in input viene salvato in un contenitore apposito di capacit\`a 1, quindi si continua a scorrere la sequenza; ad uno ad uno  si confrontano gli elementi dell'input con l'elemento contenuto nel contenitore, aggiungendo all'output il minore dei due mentre il maggiore viene salvato nel contenitore.\\
Due ulteriori algoritmi, stacksort e queuesort, si ottengono generalizzando il contenitore, in particolare sostituendolo con una struttura dati (rispettivamente una pila e una coda) di capacit\`a arbitraria.\\\\
Gli algoritmi presentati generalmente vengono iterati pi\`u volte sulla stessa sequenza. La ragione \`e che una sola iterazione non garantisce l'ordinamento completo della sequenza.\\
In questa tesi tuttavia si osserva il caso dell'applicazione di una sola iterazione di questi algoritmi ad una permutazione e delle propriet\`a che permettono di stabilire per ogni algoritmo l'insieme esatto di permutazioni ordinabili da una sola iterazione.\\
Questo tema \`e stato inizialmente trattato da Donald Knuth, in "Vol. 1: Fundamental Algorithms", \textit{The Art of Computer Programming (1968)}, che aveva individuato esattamente quali permutazioni sono ordinabili da una sola passata di stacksort. Similmente sono state trovate condizioni analoghe per gli altri algoritmi.\\\\
L'obiettivo di questa tesi \`e quello di utilizzare le condizioni note per ricercare le controimmagini di determinati pattern di permutazioni di interi per individuare condizioni che garantiscono l'ordinabilit\`a di una permutazione da parte della composizione di due algoritmi tra quelli presentati.\\\\
Per trovare questi risultati si utilizzano degli algoritmi che permettono di ricercare le controimmagini di pattern di permutazioni secondo gli algoritmi in esame, presenti in letteratura.\\
Ci si \`e avvalsi dell'utilizzo di strumenti software, realizzati appositamente, per esaminare il comportamento di questi algoritmi: uno che enumera le permutazioni ordinabili e non ordinabili da un dato algoritmo di ordinamento, insieme alle possibili immagini prodotte, e uno che mostra quali permutazioni contengano o meno un dato pattern.\\\\
Nel capitolo 2 di questa tesi vengono introdotti gli algoritmi di ordinamento in esame, nonch\'e alcuni concetti legati alle classi di pattern di permutazioni utilizzati nel corso della trattazione.\\
Nel capitolo 3 vengono presentati i programmi realizzati e ne viene spiegato l'utilizzo. Il codice sorgente di tali programmi \`e riportato interamente nell'appendice finale.\\
Nel capitolo 4 vengono mostrati i procedimenti per individuare le classi di pattern ordinabili dalle composizioni; vengono introdotti gli algoritmi per il calcolo di controimmagini e ne viene mostrata l'applicazione.\\
Nel capitolo finale si riassumono i risultati trovati e si mostra, per ogni composizione, la quantit\`a di permutazioni ordinabili di lunghezza n.
